\begin{voorstel}{Zonnepanelen op daken}
\meeschrijver{Krik van Zwet}

\begin{samenvatting}
Zonnepanelen moeten op alle bedrijfsgebouwen geïnstalleerd worden
\end{samenvatting}

\begin{multicols*}{2}

\begin{uitdaging}
Het streven is er op gericht om zonnepanelen te plaatsen op zo veel mogelijk daken van bedrijfsgebouwen. Bedrijfsgebouwen zijn meestal groter dan particuliere woonhuizen en bieden daarom plaats aan grote aantallen zonnepanelen.
\end{uitdaging}

\begin{overwegingen}
Er gebeurt al heel veel op het terrein van zonnepanelen. Naast particulieren, energiecoöperaties (postcode-roos-projecten) en overheden zijn er zeker ook bedrijven, die duurzaamheid als leidraad gebruiken om energie op te wekken middels zonnepanelen. Zonnepanelen op daken ontsieren het landschap minder, indien de zonnepanelen worden geplaatst op hoge gebouwen. Hoge gebouwen zijn in de regel bedrijfsgebouwen, zoals b.v. distributiebedrijven. Plaatsing van zonnepanelen op dit soort hoge gebouwen heeft dus twee voordelen, t.w.

\begin{enumerate}
	\item de aantallen zonnepanelen zijn groot
	\item de zonnepanelen zijn aan het oog onttrokken.
\end{enumerate}

Probleem hierbij is, dat veel bedrijven en investeerders, die dit soort gebouwen in eigendom hebben of laten bouwen, vanuit zichzelf niet overtuigd zijn van nut en noodzaak van de plaatsing van zonnepanelen.
\end{overwegingen}

\begin{aanbevelingen}
Er moet daarom een wet komen, die bedrijven verplicht om op de alle daken van hun bedrijfsgebouwen (oudbouw en nieuwbouw) vóór 1 januari 2025 zonnepanelen te installeren. Deze wet is nodig, omdat het overtuigen van de bedrijven van de noodzaak om groene energie op te wekken te lang gaat duren en ook nog eens bij te veel bedrijven niet zal lukken. Er is geen tijd meer voor eindeloos overleg.
\end{aanbevelingen}

\end{multicols*}

\end{voorstel}
