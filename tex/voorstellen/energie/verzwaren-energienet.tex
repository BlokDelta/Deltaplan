\begin{voorstel}{Verzwaren energienet}
\meeschrijver{Roelof Lanting}

\begin{samenvatting}
Tussen nu en 2030 zal er veel moeten veranderen in verband met het transport van elektriciteit, gas en waterstof (en mogelijk ook van andere energiedragers). Daarbij gaat het niet alleen om het energienet zelf (de techniek). Ook de besluitvorming over dat netwerk moet beter en - vanaf 2023 wellicht - anders. plaatsvinden.

Bij het energienet gaat om een zeer belangrijke voorwaarde om in 2030 energieneutraal te zijn. De bijdrage aan dat resultaat per 2030 en de politieke verantwoordelijkheid daarvoor zijn het onderwerp van dit voorstel.

Voor een CO2-neutrale energiesituatie in 2030 moet het nu bestaande elektriciteitsnet aangepast en ook uitgebouwd worden.

Als de netbeheerders er per eind 2022 niet toe bereid en in staat lijken om dat te doen, zullen er wijzigingen moeten komen in hun positie en in de regelingen waarop die positie berust.
\end{samenvatting}

\begin{uitdaging}
In 2021 en 2022 de beslissingen nemen die maken dat de ombouw van het elektriciteitsnet per 2030 gerealiseerd kan zijn..
\end{uitdaging}

\begin{overwegingen}
Het elektriciteitsnet heeft twee hoofdroutes, namelijk hoogspanning, ingericht door het Duits-Nederlandse staatsbedrijf Tennet en het net voor midden-/laagspanning. Daarvoor zijn drie netbeheerders verantwoordelijk : Enexsis, Liander en Stedin, elk in een aantal provincies. Deze bedrijven zijn monopolist en mogen geen andere bedrijvigheid ontplooien. Dit alles berust op de Elektriciteitswet. Naast leidingen zijn b.v ook verdeelstations onderdeel van die netten.

Het elektriciteitsnet (eigenlijk : de netten) is primair opgezet om in centrales (fossiel) opgewekte elektriciteit te transporteren.. Om daarnaast of in plaats daarvan ook decentraal uit wind en zon opgewekte elektriciteit blijkt het net nu al ontoereikend.

Voor de situatie in 2030 waarin volledig op zon “geleefd” wordt, zal het net dan ook aangepast moeten worden. Ook het omzetten van duurzaam opgewekte stroom in waterstof voor industrieel gebruik vereist aanpassingen.
Hoewel het al jaren duidelijk is dat deze aanpassingen nodig zijn, dralen de netbeheerders om die uit te voeren. Hun rol is de facto remmend.

In de CO2-neutrale situatie van 2030 wordt meer elektriciteit gebruikt dan nu, onder andere doordat het gasloos vervaardigen van goederen in veel situaties meer elektriciteit vergt.
Energiezuinigheid/besparen en de opwekking van stroom door inwoners/bedrijven zelf zullen die toename waarschijnlijk niet kunnen opvangen. Het net moet dus ook worden aangepast om meer elektriciteit van A naar B te brengen.

De sterke toename van het verbruik van stroom door datacentra compliceert de situatie ook nog eens (NOOT : zie NRC 22-6-2020, Dat verbruik overtreft dat van middelgrote steden. Hierover meer in Delta-voorstel … - titel - …. EINDE NOOT)

Dat de energieneutraliteit technisch grote veranderingen in de toestand, de capaciteit en de werking van het net nodig maken, staat vast.

Meer decentraal opwekken, gesloten systemen waterstof en andere (opslag-)technieken kunnen helpen, maar niet voldoende.

Welke veranderingen nodig zijn, is nog niet volledig duidelijk. Maar het zal sowieso veel kosten. . Netwerk- bedrijven noemen percentages (in 2030 wel 207 \% van de kosten in 2015) maar op basis van het huidige totale verbruik. En dat zal in 2030 groter zijn. Ze benadrukken hoe moeilijk het zal zijn en proberen geld op te halen, o.a. bij provincies.

Urgenda noemt in “Vijf keer anders” - niet onderbouwd - tot 2030 jaarlijks 2 \% van het BNP. Wat ook ontbreekt, is een onafhankelijke doorrekening. Die moet er in 2022 wel zijn.

Op het niveau van burgerinitiatieven is ondertussen al veel verduurzaming in gang gezet. . Industriële grootverbruikers voeren aanzienlijke veranderingen door moeten dat ook. Beide moeten zich kunnen baseren op heldere gegevens mbt een adekwaat net in 2030 en daarna.

Sinds 2004 (liberalisering, op aansturen van Europa) bouwen, beheren en exploiteren de net- beheerders het elektriciteitsnet, dus Tennet dat voor hoogspanning en Liander, Enexis en Stedin die voor midden- en laagspanning.

In de Elektriciteitswet komen mogelijkheden tot aansturing door de Minister van Economische Zaken voorkomen. Echter, de manier waarop de overheid tot nu toe haar bevoegdheden en positie in de praktijk gebruikt ten opzichte van de monopolistische bedrijven blijkt onvoldoende om de “omslag naar neutraal” te laten plaatsvinden. Mogelijk moeten die bevoegdheden ook worden uitgebreid. Maar omdat wijzigingen in organisatiestructuren en in wetgeving vertragend werken, kiezen we daar nu nog niet voor. Ze moeten alleen dan plaatsvinden als dat eind 2021 nodig en/of onvermijdelijk is gebleken. 

De liberalisering heeft teveel ruimte gecreëerd om niet te doen wat klaarblijkelijk nodig is. Daarom zal de overheid al meteen in 2021 een grotere rol moeten nemen en gebruik maken van bestaande bevoegdheden gebruiken.
\end{overwegingen}

\begin{aanbevelingen}
In de jaren 2021 en 2022 worden de mogelijkheden in de Elektriciteitswet  maximaal gebruikt om concreet te maken wat er technisch moet veranderen om in 2030 een elektriciteitsnet te krijgen dat duurzaam opgewekte elektriciteit kan transporteren naar waar het gebruikt wordt.

In 2022 wordt door het PBL een doorrekening  van de kosten van aanpassing van het net/de netten uitgevoerd, waarbij koppeling van gebruik voor waterstof en gas en andere opties meegenomen worden.

In de 2023 en de eerstvolgende jaren moeten de noodzakelijke veranderingen in het net (de netten) uitgevoerd worden. Het zal van de per eind  2021 gebleken handelingsbereidheid van de netbeheerders afhangen of zij in de laatste acht jaar tot 2031 in dezelfde positie als nu kunnen blijven.. Als dat niet het geval is , zal er in 2022 wijziging van de wetgeving moeten plaatsvinden.
\end{aanbevelingen}

\paragraph{Literatuur}
www.hoogspanningsnet

Vijf keer anders, uitgave van Urgenda

NRC 10-10-2019 (p. E6/7) en 14 mei 2020 (p. E 3)

\end{voorstel}
