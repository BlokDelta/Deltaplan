\begin{voorstel}{Wind op zee}
\meeschrijver{Rens Baardman}

\begin{samenvatting}
Verdriedubbeling van ambitie zorgt voor schone energie die een toevoeging is voor de Noordzee-natuur en waar burgers ook financieel van profiteren
\end{samenvatting}

\begin{uitdaging}
Momenteel brengen de windmolens op de Noordzee 1 GW aan elektriciteit op. Met de huidige plannen wordt dit ruim 11 GW in 2030, 12\% van het huidige elektriciteitsverbruik. Een mooi aandeel, maar er is nog veel meer potentie. Zeker als de vraag naar stroom toeneemt door elektrificatie van de industrie en auto’s, is wind op zee cruciaal om op een duurzame manier aan die vraag te voldoen. Hoe kunnen we meer elektriciteit opwekken uit windenergie op zee? Hoe zorgen we dat de natuur en individuele burgers kunnen profiteren van de parken?
\end{uitdaging}

\begin{overwegingen}

\paragraph{Windparken}
Een windpark op zee bestaat uit tientallen tot honderden windmolens met een capaciteit van 5 à 10 MW per stuk. De windmolens zijn via lange kabels op de bodem van de zee verbonden met het hoogspanningsnet aan de wal. De opgewekte stroom kan met grote omvormers op zee worden omgezet van wissel- naar gelijkstroom voordat het naar land wordt getransporteerd.  Er zijn voorstellen om kunstmatige eilanden te bouwen waar de omvormers kunnen staan, bijvoorbeeld bij de Doggersbank, een ondiepte ver in de Noordzee. Hier leggen dan netbeheerders uit Duitsland, Denemarken en Nederland samen een internationaal knooppunt voor windenergie en waterstof aan \parencite{north_sea_wind_power_hub_north_2019}. Zo’n ‘energie-eiland’ zou ook dicht bij de Nederlandse kust kunnen worden gebouwd.. Net als de Marker Wadden is zo’n eiland een kennis- en excursiecentrum met bezoekers en een bijzondere plek voor natuur en biodiversiteit.

\begin{infobox}{Wat betekent ‘capaciteit’?}
De capaciteit van een windmolen duidt op het maximaal op te wekken hoeveelheid stroom als er veel wind staat. De behaalde opbrengst is afhankelijk van de daadwerkelijke windsterkte. Gemiddeld over een jaar behaalt een windmolen zo’n 50\% van de maximale capaciteit \parencite{lensink_kosten_2017}. Dat levert op jaarbasis bijna 44 GWh op, genoeg elektriciteit voor 16.000 huishoudens.
\end{infobox}

\paragraph{Planning en aanleg}
Vanaf het eerste initiatief kan het tot wel tien jaar duren voordat een windpark is aangelegd \parencite{westra_offshore_2014}, en kost honderden miljoenen euro’s. Het proces begint met de tenderfase waarbij de overheid kavels selecteert waar geïnteresseerde bedrijven een voorstel voor kunnen doen. Daarna volgt ontwerp, aanleg en aansluiting. Vanaf dan kan het park zeker 20 jaar rendabel draaien \parencite{lensink_kosten_2017}. Sinds 2018 zijn alle tenders subsidieloos. Alleen de netaansluiting naar het land is voor de rekening van de netbeheerder, en daarmee voor de staat.

\paragraph{Locaties op de Noordzee}
Nieuwe windparken kunnen worden gebouwd binnen de Exclusieve Economische Zone: een gebied dat anderhalf keer zo groot is als Nederland, dat Nederland mag exploiteren. Windparken worden niet aangelegd in de gebieden die nodig zijn voor scheepvaartroutes, zandwinning, visserij, militaire terreinen en natuurgebieden. Ook moet de zee niet te diep zijn: vaak wordt 30 meter als maximum gehanteerd. Daarnaast wil je de parken het liefst dicht bij de kust hebben staan, zodat de elektriciteitsverbinding naar de wal niet te duur wordt, en niet te dicht bij de kust, zodat je de parken niet ziet. Alsnog zijn er nog veel geschikte gebieden over. Het Planbureau voor de Leefomgeving schetste al een scenario waarbij in 2050 voor 60 GW aan Nederlandse windmolens is geïnstalleerd \parencite{matthijsen_toekomst_2018}.

\paragraph{Natuur}
De aanleg van windparken is een kans voor natuur op de Noordzee. Tussen windmolens wordt niet gevist, waardoor de zeebodem, riffen en daarmee biodiversiteit kunnen herstellen.  De fundering van windmolens zijn ook goede aanhechtingsplaatsen voor oesters en ander bodemleven. Er lopen experimenten om in windparken kunstriffen te maken \parencite{didderen_offshore_2019}. Ook opent zich een weg naar ‘zeeboerderijen’ (aquacultuur): tussen de windmolens kunnen krabben, kreeften, oesters, mosselen en zeewier worden gekweekt.
Belangrijk is dat (bruin)vissen, zeehonden, vogels en vleermuizen zo min mogelijk last hebben van de molens. Dat kan met geluiddempende bellenschermen om windmolens heen tijdens de aanleg, nieuwe hei-technologie die minder lawaai maakt, en het preventief uitzetten van turbines als er vogels of vleermuizen worden gedetecteerd.

\paragraph{Burgerparticipatie}
De Noordzee en de wind die daar waait zijn ons collectief natuurlijk kapitaal, dus niet alleen grote ontwikkelaar, maar juist alle burgers hebben recht op het dividend daarvan. Burgers, energiecoöperaties, pensioenfondsen en lokale overheden moeten dus mee profiteren. Burgerparticipatie is een belangrijk onderdeel van het Klimaatakkoord, maar is nog geen criterium bij de tenderprocedures. Bij wind op land is burgerparticipatie al veel gebruikelijker \parencite{nwea_gedragscode_2016}.

\end{overwegingen}


\begin{aanbevelingen}

\speerpunt{In 2030 staat er op de Noordzee voor 30 GW aan windmolens.}
Dat is een verdriedubbeling van de huidige ambitie in het Klimaatakkoord. Die versnelling wordt gerealiseerd door het rap uitgeven van grote, nieuwe kavels. Door het stimuleren van technische ontwikkelingen stijgt de capaciteit per windmolen. Een stevige CO2-belasting maakt windprojecten nog aantrekkelijker voor ontwikkelaars, zodat kavels straks geveild kunnen worden. Die opbrengsten gaan in een nationaal Klimaatfonds en worden gebruikt om nieuwe parken te financieren.

\speerpunt{Burgerparticipatie wordt een voorwaarde} in de tenderprocedure, zodat burgers bij kunnen dragen aan minstens 20\% van de investeringen voor een park.
Dat kan bijvoorbeeld via het Klimaatfonds of met lokale energiecoöperaties. Zo worden alle Nederlanders mede-eigenaar van onze stroomvoorziening.

\speerpunt{Windmolenparken worden een aanwinst voor de Noordzee-natuur.}
Onderzoek naar herstel van zeebodem en -leven wordt gestimuleerd. Zeeboerderijen zijn een alternatief voor de huidige visserij.

\speerpunt{Kunstmatige energie-eilanden} zijn nieuwe natuurgebieden, met faciliteiten voor recreatie, een haven voor pleziervaart, en kenniscentra van waaruit (school)excursies worden georganiseerd.
We laten zien waar onze energie vandaan komt, maken duidelijk hoe de transitie er fysiek uitziet, en het creëert trots op onze duurzame infrastructuur.

\end{aanbevelingen}

\end{voorstel}