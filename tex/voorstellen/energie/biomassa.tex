\begin{voorstel}{Biomassa}
\meeschrijver{Renaat van Rompaey}

\begin{samenvatting}
Tegen 2030 gebruiken we alleen nog biomassa en hout dat duurzaam geproduceerd is én hergroeit.  Het gebruik ervan wordt schoon en veilig.  Ook in Nederland zorgt Groenlinks tegen 2030 voor twee keer zoveel bos en bomen.
\end{samenvatting}

\begin{multicols*}{2}

\begin{uitdaging}
Momenteel leidt het gebruik van biomassa vaak tot natuurvernietiging en niet-duurzaam bosbeheer (Cardellichio et al. 2010). Het is belangrijk dat we het gebruik weten te beperken tot de hoeveelheid die duurzaam geproduceerd kan worden. In Nederland kunnen we twee maal zoveel hout en biomassa produceren als we op dit moment doen en tegelijk genieten van de vele bijkomende voordelen zoals waterberging, zuivere lucht, klimaatkoeling, schaduw en habitat voor biodiversiteit.

Vuur en rook zijn ook bronnen van luchtvervuiling, brengen gevaar met zich mee en kunnen schadelijk zijn voor de gezondheid, hoewel de mens er al een miljoen jaar mee omgaat. Samen rond een houtvuur brengt mensen opnieuw in contact met elkaar, vlammen zijn mooi om naar te kijken, directe warmte is behaaglijk, het creëert gezelligheid.
\end{uitdaging}

\begin{overwegingen}
\paragraph{Vuur}
Sinds meer dan één miljoen jaar heeft de mens geleerd het vuur te beheersen en gebruiken. Als brandstof gebruikte hij gedroogde plantenresten en hout. Nog tot op de dag van vandaag is gedroogde biomassa de belangrijkste brandstof voor het grootste deel van de wereldbevolking en het duurzaam gebruik ervan één van de grootste uitdagingen van de groene transitie.

\paragraph{Hout}
Hoewel biomassa kan slaan op om het even welk weefsel van natuurlijke oorsprong (mest, slib, voedselresten, dierlijke vetten, slachtafval, landbouwafval, zagerijafval; Kamerstuk 35 167, 22-11-2019,), beslaat hout het merendeel van de biomassa die we hier bedoelen. Hout geeft stevigheid aan planten en laat ze toe honderd meter hoog te groeien en duizenden jaren oud te worden. Voor het stenen tijdperk was er al een ‘houten tijdperk’, toen de mens leerde van hout en vezels allerlei gebruiksvoorwerpen (wapens) en behuizing te maken. Hout is onze belangrijkste grondstof en is, na olie, het belangrijkste wereldhandelsproduct (wto.org).

\paragraph{Duurzaam beheerd bos}
Het meeste hout groeit in bossen. Groenlinks is voor duurzaam bosbeheer. Niet alleen groeit er in duurzaam beheerde bossen evenveel hout bij als er gekapt wordt, maar daarnaast vervult het bos er ook haar andere functies, zoals habitat/thuis voor biodiversiteit, bodemvormer, waterberger in de watercyclus, luchtzuiveraar, klimaatkoeler, bron van voedsel en medicijnen, biotoop (leuke plek om te zijn) voor de mens (en zijn huisdier). Het bos in al zijn verschijningsvormen en gradaties is onmisbaar voor de mens. Erbuiten ligt alleen de steppe, de woestijn, het hooggebergte, de toendra en de poolstreken, waar het te droog, te heet, te zout of te koud is voor bomen om te groeien.

Groenlinks is voor bosbehoud. De mens heeft al te veel ontbost, voor landbouw, mijnbouw, infrastructuur, steden en bebouwde oppervlakte, wegen, industrie. Daarnaast heeft de mens bijna alle bosecosystemen op de wereld overbenut, overgebrand, en overgeëxploiteerd. Groenlinks wil dat de ontbossing stopt en dat bosherstel plaatsvindt. Bossen kunnen herplant worden, beschermd worden, duurzaam beheerd worden. In Nederland en buitenland is Groenlinks voor natuurherstel, natuurbouw soms, en natuurbescherming, door natuureducatie, multifunctioneel landgebruik en respect voor het leven op aarde en de natuurlijke processen die onze landschappen vormgeven tot ware kunstwerken van de natuur. De levende wezens zijn afhankelijk van elkaar voor hun voortbestaan en leven in symbiose op deze aardbol.

\paragraph{Biomassa als product van duurzaam bosbeheer}
Hout is een prachtig natuurproduct en wordt al te vaak vervangen door metaal, plastic of beton. We moeten zuinig zijn op hout en bosproducten die duurzaam geproduceerd zijn. Hergebruik en recyclage verdienen de voorkeur boven verbranding.

Om onze hele energievoorziening op biomassa te laten lopen, zijn enorme oppervlakten bos nodig. De ruimte is echter schaars op plekken waar voldoende regen valt opdat er bos zou groeien. Op diezelfde plekken willen we aan landbouw en veeteelt doen voor onze voedselvoorziening. Bij gebrek aan ruimte in Nederland, halen we het uit het buitenland.

In het buitenland gelden andere, soms minder strenge wetten en zo zien we dat er in het buitenland veel natuur verdwijnt of bos niet-duurzaam beheerd wordt om aan de Nederlandse vraag te voldoen. Groenlinks is voor eerlijke wereldhandel, met respect voor de natuur. Importeurs  moeten dit aantonen zodat de stromen niet-duurzaam geproduceerde producten verkleinen, wat dan weer de prijs kan verhogen.

\paragraph{Duurzaam, maar ook te duur om op te stoken}
En zo komen we tot de slotsom: duurzaam geproduceerde biomassa is duur en zal dus geen mirakeloplossing zijn om Nederland klimaatneutraal te maken. Er zijn andere goede redenen om in Nederland meer bomen te krijgen en meer bos te hebben. Groenlinks is voor natuurlijke bosverjonging met inheemse soorten. Dat levert robuustere en gevarieerde ecosystemen op, beter bestand tegen wind, vuur, water, vraat en zelfs het nieuwe klimaat. Tien procent bosbedekking in Nederland is echt te weinig. Wij willen twintig procent bosbedekking in 2030.

Bij verzagen van boomstammen naar planken houd je maar dertig procent van het oorspronkelijke volume over. De houtverwerkende industrie heeft dus een hoop afvalhout en - zaagsel en dat moet zo nuttig mogelijk hergebruikt worden. Verbranding voor terugwinning van de opgeslagen energie zou de laatste optie moeten zijn.

\paragraph{Biomassa in de energietransitie: alleen goede biomassa, als het niet anders kan, en zonder subsidie}
Voor zover biomassa in de energietransitie gebruikt moet worden, moet het goede biomassa zijn, ongesubsidieerd, en voor hoogwaardige toepassingen of piekvermogen. Er moet dus gecontroleerd worden of de biomassa niet leidt tot ontbossing, verslechterde bodem- of waterkwaliteit, verlies van biodiversiteit, verminderen van voedsel- en waterzekerheid, of schendingen van mensenrechten. Daarnaast moet er een duidelijk positief klimaateffect zijn, bijvoorbeeld bij piekvermogen minstens 70\% minder CO2-uitstoot dan aardgas binnen een bosherstelperiode van tien jaar. Biomassa die niet aan deze criteria voldoet, mag niet in Nederland gebruikt worden. Biomassa die wel aan deze criteria voldoet, mag gebruikt worden, maar alleen voor hoogwaardige toepassingen of piekvermogen. Stroomopwek met biomassa is dus alleen een optie als er niets anders beschikbaar is, bijvoorbeeld in enkele windstille winterweken per jaar. Warmteketels leveren basislast en deze mogen dus niet meer op biomassa draaien.\footnote{Zie ook https://www.natuurenmilieu.nl/wp-content/uploads/2020/07/Biomassa-Visie-2020.pdf}

\paragraph{Schoon stoken}
Het stoken van hout in de open haard brengt veel fijnstof in de lucht. De Gemeente Den Haag heeft \href{https://www.denhaag.nl/nl/in-de-stad/natuur-en-milieu/duurzaamheid/verstandig-hout-stoken.htm}{9 tips om zo schoon mogelijk te stoken.} Zo gaan de tips onder andere over het aanschaffen van de juiste kachel, aanmaakmethode en een schone schoorsteen. Bij ongunstige weersomstandigheden, zoals weinig wind en mist, blijft rook langer hangen. Met een stookalert roept het RIVM mensen op om dan geen hout te stoken.

\paragraph{Veilig barbecueën}
Bij verbranding en verkoling van biomassa voor de barbecue komen verschillende kankerverwekkende stoffen (zoals heterocyclische amines en polycyclische koolwaterstoffen) vrij je beter niet opeet of aan je vingers krijgt. De \href{https://www.kanker.be/alles-over-kanker/aantoonbaar-risico/tips-om-gezonder-te-barbecue-n}{Stichting tegen kanker heeft tips voor een gezonde barbecue.} Groenlinks promoot zelfredzaamheid. \href{https://www.kanker.be/alles-over-kanker/aantoonbaar-risico/tips-om-gezonder-te-barbecue-n}{Kamperen is de mooiste zomersport,} waardoor je steeds maar jonger wordt. Weten hoe je in de natuur (over)leeft, hoort ook bij een ecologische levenswijze, dus wonen, zich verwarmen, koken (ook water koken) in de natuur, hoort daarbij.

\end{overwegingen}

\begin{aanbevelingen}
\speerpunt{Tegen 2030 gaan we duurzaam met biomassa om. De bossen waaruit we biomassa winnen, worden duurzaam beheerd.}
Er komt op de wereld veel biomassa vrij bij ontginningen en landgebruikconversie, maar ook bij laan- of fruitbomen die vervangen worden. Als daarvoor de correcte vergunningen afgegeven zijn, hoeft dat hout niet ‘fout’ te zijn en is zorgvuldig gebruik ervan, geen probleem. Maar elk lucratief gebruik kan leiden tot overexploitatie, dus de overheid moet zijn rol als regulator vervullen.

\speerpunt{Elektriciteitsproductie met biomassa kan alleen bij garantie dat het duurzaam geproduceerd is én hergroeit.}
Voor hout en dus ook pellets geldt dat  het legaal geproduceerd moet zijn (EU Timber Regulation) om ingevoerd te kunnen worden in de EU. Nu is legaliteit nog geen garantie voor duurzaamheid, maar het verwijst wel dat er rekening gehouden is met de rurale bevolking of inheemse volkeren en dat de natuurbeschermingswetten niet geschonden zijn.

Jong, hergroeiend bos legt veel CO2 vast en als de Nederlandse subsidie verstrekt wordt onder veronderstelling van CO2-vastlegging van alle uitstoot van die centrales, dan moet dat ook zijn gecertificeerd. 

\speerpunt{Elektriciteitsproductie met biomassa wordt alleen gebruikt wanneer er langdurig geen andere duurzame energie voorhanden is.}
Dat kan bijvoorbeeld in een windstille winterweek zijn.

\speerpunt{We gebruiken alleen goede biomassa.}
Dat betekent dat alleen biomassa die niet leidt tot ontbossing, verslechterde bodem- of waterkwaliteit, verlies van biodiversiteit, verminderen van voedsel- en waterzekerheid, of schendingen van mensenrechten, gebruikt mag worden. Ook moet de biomassa binnen tien jaar bosaangroeitijd minstens 70\% klimaatvriendelijker zijn dan aardgas.

\speerpunt{Biomassa heeft geen subsidie meer nodig.}
De huidige subsidies worden uitgefaseerd. De nieuwe, goede biomassa hoeft niet gesubsidieerd te worden, omdat deze alleen wordt ingezet voor hoogwaardige toepassingen of piekproductie van elektriciteit, die mede dankzij de gestegen CO2-belasting toch al hoge opbrengsten geeft.

\speerpunt{Biomassa gebruiken we niet voor warmteopwek.}
Opwek van stadswarmte is laagwaardig gebruik, en biomassa is hier te schaars voor. Subsidies voor warmteopwek met biomassa worden dan ook stopgezet en nieuwe warmteprojecten mogen geen biomassa meer voor basislast warmteopwek gebruiken.

\speerpunt{Aanzienlijke herbebossing en bosherstel tegen 2030 houden Nederland leefbaar bij klimaatverandering en herstelt onze waterbalans, zuivert onze lucht, verkoelt en verrijkt de biodiversiteit.}
Nederland is veel te kaal. Langs wegen, waterlopen en langs perceelsgrenzen kan veel bos, struweel en houtkanten verschijnen. Groenlinks wil een boomrijke leefomgeving met twee keer zoveel bomen in 2030 binnen en buiten de bebouwde kom. De bebouwde kom moet ontstenen en ontharden. Na de bloemrijke wegbermen verschijnt er ook een bloem- en boomrijke binnenstad met groendaken. Subsidies voor monocultuur-landbouw moeten omgebogen worden naar landschapszorg en kleinschalige voedselproducerende gemengde systemen.

\end{aanbevelingen}

\paragraph{Literatuur}
Cardellichio, P.; et al. (2010). Walker, T. (ed.). "Massachusetts Biomass Sustainability and Carbon Policy Study: Report to the Commonwealth of Massachusetts Department of Energy Resources" (PDF). Natural Capital Initiative Report NCI-2010-03. Brunswick, Maine.: Manomet Center for Conservation Sciences.

Kamerstuk 35 167, 22-11-2019, Regels voor het produceren van elektriciteit met behulp van kolen (Wet verbod op kolen bij elektriciteitsproductie),

V. Cram Martos \& F. Romig. Trade in energy and forestry, a perspective from the United Nations Economic Commission for Europe. Contribution from the United Nations Economic Commission for Europe

https://www.natuurenmilieu.nl/wp-content/uploads/2020/07/Biomassa-Visie-2020.pdf

De Dagelijkse Standaard, 20 mei 2020. Jesse Klaver claimt dat GL tegen biomassa is, wordt voor paal gezet met allerlei pro-biomassa GL-wethouders. https://www.dagelijksestandaard.nl/2020/05/jesse-klaver-claimt-dat-gl-tegen-biomassa-is-wordt-live-voor-paal-gezet-met-allerlei-pro-biomassa-gl-wethouders/

Minister Wiebes, 17-10-2019. Memorie van antwoord wetsvoorstel Wet verbod op kolen bij elektriciteitsproductie. https://www.rijksoverheid.nl/documenten/rapporten/2019/10/17/regels-voor-het-produceren-van-elektriciteit-met-behulp-van-kolen-wet-verbod-op-kolen-bij-elektriciteitsproductie

35 167. Regels voor het produceren van elektriciteit met behulp van kolen (Wet verbod op kolen bij elektriciteitsproductie)

\url{https://nl.wikipedia.org/wiki/Menselijke_vuurbeheersing}

\end{multicols*}

\end{voorstel}
