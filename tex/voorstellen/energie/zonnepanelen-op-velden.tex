\begin{voorstel}{Zonnepanelen op velden}
\meeschrijver{Edo van Baars}

\begin{samenvatting}
Voorstel: zonneparken kunnen zorgvuldig worden ingepast in landelijk gebied door compensatie eis te stellen, ingevuld door omwonenden, om weerstand te voorkomen
\end{samenvatting}

\begin{uitdaging}
In het Nationaal Klimaatakkoord is afgesproken dat in 2030 35 TW aan elektriciteit op land worden opgewekt. Ter illustratie: dat is vier keer de hoeveelheid energie dat de haven van Rotterdam gebruikt. Dat bestaat uit wind op land en grootschalige zon-pv  (zonnepanelen, in totaal >15 kWp)[i], oftewel op velden en op grote daken. Kleinschalige zon-op-dak (<15 kWp) is hierin niet meegerekend.

De komende jaren zullen er nog aardig wat zonnevelden/zonneparken (termen worden hier door elkaar gebruikt) worden aangelegd in het buitengebied, o.a. omdat in de Regionale Energie Strategieën (RES)[ii] de nadruk meer op zon- dan op windenergie leggen, vanwege de verwachte weerstand t.o.v. windmolens. En omdat zonneparken commercieel nóg aantrekkelijker zijn dan grootschalige opwek op daken, o.a. de aanleg is goedkoper is in het veld dan op het dak.[iii]

We staan achter de ambities in het Klimaatakkoord maar kunnen er niet omheen dat de zonneparken een ruimtelijke impact hebben op het buitengebied. \textbf{\em{We zien zonneparken meer als noodzaak, niet als droombeeld. Laten we die impact daarom zoveel mogelijk beperken.}} Het Nederlandse landschap staat al zo onder druk. En dan ligt er ook nog een grote woningbouwopgave in het verschiet (bouw van ca. 1 miljoen nieuwe woningen voor 2030).[iv]

De uitdaging met zonnevelden zit ‘m niet in het gebrek aan animo om ze aan te leggen, het aantal parken is de afgelopen twee jaar bijna verdubbeld. Ze worden steeds groter, voorheen gemiddeld 2 hectare en nu 20.[v] De uitdaging is de ruimtelijke inpassing.

Bij windmolens zie je vaak weerstand van omwonenden. Om een vergelijkbare weerzin te voorkomen, is het belangrijk dat toekomstige zonneparken goed worden ingepast in het landschap. De kans dat weerstand toeneemt is reëel gezien een (groot) zonnepark een industriële uitstraling heeft, dat contrasteert met natuur/landbouwgrond[vi].

Het kabinet heeft al eerste stappen gezet om de groei van zonneparken meer te reguleren, met een nieuw toetsingsinstrument: de Zonneladder. Deze zal worden vastgelegd in de Nationale Omgevingsvisie en de RES, en is door gemeentes en provincies te gebruiken. De Zonneladder geeft een voorkeursvolgorde aan voor zonnevelden[vii]:
\begin{enumerate}
	\item Daken en gevels van gebouwen;
	\item Bebouwd gebied, onbenutte terreinen zoals parkeerplaatsen, geluidsschermen langs de weg;
	\item Landelijk gebied: natuur-en landbouwgebieden zijn niet uitgesloten, maar de voorkeur ligt bij locaties zoals waterzuiveringsinstallaties, vuilnisbelten, binnenwateren of bermen van spoor- en autowegen.
\end{enumerate}

Deze voorkeursvolgorde betekent niet dat er geen zon-pv meer in het landelijk gebied zal worden ontwikkeld. Want de RES eist 35 TW opwek, en daarbij telt kleinschalige zon-op-dak energie - geschat op 7 TW in 2030 -  niet mee. De verwachting is dat het volume grootschalig zon-pv toeneemt van 1,5 TW in 2018 naar ca. 8,5 TW in 2030. Hiervan is 74\% een dak-opstelling, \textbf{\em{26\% een veldopstelling}} en 0,2\% een drijvende installatie in het water. Op basis van de gerealiseerde productie en projecten die in de pijplijn zitten, wordt in 2030 naar verwachting 26 TW hernieuwbare energie door grootschalig zon en wind op land opgewekt. Dit betekent dat de resterende opgave circa 9 TW bedraagt.[viii]

Het is aannemelijk dat niet alle daken komen vol te liggen met panelen omdat particuliere eigenaren niet mee (willen) doen. Tevens is de ruimte in bebouwd gebied zeer schaars en een kleinschalig zonnepark (kleiner dan 2 MWh) is vaak niet rendabel. Daarom zal er dan naar het buitengebied worden gekeken.

Wij gaan mee in de gedachte van de Zonneladder (en de RES doelstellingen). \textbf{\em{Maar we komen met een aanvulling (zie hoofdstuk ‘voorstel’) op de Zonneladder om ruimtelijke inpassing beter te regelen en weerstand van omwonenden zoveel mogelijk te voorkomen. Voor een zo soepel en snel mogelijke energietransitie, met als tijdshorizon 2030.}}
\end{uitdaging}

\begin{overwegingen}
De volgende overwegingen hebben meegespeeld in de totstandkoming van ons voorstel.

\paragraph{Ruimtelijke inpassing}

\begin{itemize}
	\item \underline{Zon oriëntatie, hellingshoek, afstand tussen panelen}: de meeste zonneparken hebben een zuid-opstelling, in Nederland 88\%. De hellingshoek is tussen de 20-45 graden. Er is ruimte tussen de panelen en dus mogelijkheden voor andere functies. De oost-west opstelling heeft een wat lagere hellingshoek en de panelen staan als een dakconstructie. Er is geen ruimte tussen de panelen, deze opstelling kan wel compacter gebouwd worden dan de zuid-opstelling. Met de zon meedraaiende panelen zijn niet voorhanden op commercieel rendabele schaal.[ix]

	\item \underline{Omvang:} een klein park (<2 MW, oftewel 4 hectare[x]) kan direct op het net worden aangesloten. Een groot park (>2 MW) heeft meer ruimtelijke impact en behoeft daarnaast speciale aansluiting: een trafostation.

	\item \underline{Vorm en patroon:} het park kan aangepast worden aan bestaande landschappelijke elementen zoals dijken, watergangen, houtwallen, paden of bomenrijen. Hierdoor kan het strakke patroon van het park doorbroken worden.

	\item \underline{Andere bijkomende voorzieningen:} naast de zonnepanelen zelf kent een zonnepark een aantal bijkomende elementen zoals de aansluiting, omvormers, trafo’s, inkoopstation, camerasystemen en eventuele hekwerken en toegangspoorten. De ruimtelijke impact kan verkleind worden door te kiezen voor bescheiden kleuren voor de bijkomende infrastructuur. Helaas kunnen zonnepanelen zelf niet goed functioneren met een andere kleur dan zwart/blauw, dat leidt namelijk tot een groot rendementsverlies.

	\item \underline{Begrenzing:} verzekeraars vereisen dat zonneparken kunnen worden afgesloten tegen vandalisme. Dit hoeven niet per se hekken te zijn. Stevig, opgroeiend groen (hagen, houtwallen, bomenrij, struweel, riet) en of brede watergangen kunnen worden gebruikt als begrenzing. Een groene afscheiding zorgt voor minder zichtbaarheid van het park, verkleint de ruimtelijke impact. Daarnaast geeft krijgt het landschap meer diepte, het biedt een spannender en afwisselender (coulissen-landschap) aanzicht dan een leeg, ruim verkaveld landschap. Het wordt aantrekkelijker om te recreëren.

\end{itemize}

\paragraph{Natuur, biodiversiteit en landbouw}
Vanuit de hoek van natuurorganisaties en agrarische belangenorganisaties komt veel kritiek op zonnevelden. De panelen bedekken de bodem waardoor en geen regenwater en zonlicht bij kan, waardoor deze verschraald. De bodem is dan minder of niet bruikbaar voor natuur-/landbouwdoeleinden, wanneer een zonnepark eventueel weer wordt opgeheven. Het windmolens nemen op de grond minder plaats in waardoor agrariers er gemakkelijk nog hun bedrijf onder kunnen runnen. De belangengroepen betreuren het massale omzetten van agrarische grond naar zonnevelden. Echter kiezen (met name stoppende) boeren hier zelf voor, temeer omdat het jaarlijks rendement hoger ligt dan met agrarisch gebruik.[xi]

Zonneparken hoeven niet alleen nadelig te zijn voor natuur en biodiversiteit, \textbf{\em{ze kunnen ook een kans zijn voor het landschap.}} Door grootschalige ruilverkaveling medio 20e eeuw zijn kleine kavels opgebroken en natuurlijke afscheidingen – zoals houtwallen – weggehaald. Juist houtwallen bieden beschutting voor vogels, egels, vossen en insecten, ze functioneren ecologische infrastructuur. \textbf{\em{Zonneparken breken de grote, lege kavels weer op, en kunnen deze houtwallen weer terugbrengen, mits dat een vereiste wordt – zie voorstel.}} Het gesloten karakter van het park zorgt ervoor dat er nauwelijks mensen en vervoersmiddelen komen – hoogstens monteurs en hun auto’s – waardoor dieren vrij spel hebben. Indien ze toegang hebben tot het gebied en er niet intensief gemaaid wordt.[xii] De keuze voor welke soort begrenzing (eerder besproken) wordt toegepast – natuurlijk of kunstmatig – is hierbij van groot belang.

\paragraph{Combinatie met andere functies}
Een groene afscheiding is positief voor de natuurwaarde. Ook binnen het zonnepark kan de natuurwaarde worden verbeterd. Zeker in het geval van grond waar intensieve landbouw werd verricht: de biodiversiteit is daar de laatste decennia teruggelopen door o.a. gebruik van pesticiden. Bij voldoende ruimte (zuid-opstelling) tussen de panelen kan de biodiversiteit worden versterkt, bijvoorbeeld wanneer er mossen kunnen groeien, die goed met schaduw kunnen omgaan. Regenwater kan ondanks de panelen alsnog de bodem bereiken, de grond droogt niet uit, het bodemleven krijgt meer kans. uit onderzoek in het Verenigd Koninkrijk bleek dat het gemiddeld vochtgehalte niet lager lag in een zonnepark dan in op normaal grasland.[xiii]

Bij nog meer ruimte tussen de panelen kunnen er schapen en kippen worden gehouden. Daarnaast is er de mogelijkheid voor het plaatsen van insectenhotels en nestkasten, zoals is gedaan bij zonnepark De Kwekerij te Hengelo (Gelderland). Het graven van kleine poelen tussen en naast panelen kan ook andere flora en fauna trekken. Wanneer er looppaden zijn, kan er een educatiefunctie aan worden gekoppeld.[xiv]

\end{overwegingen}

\begin{aanbevelingen}
Naar aanleiding van de ontwikkelingen die spelen op het gebied van zon-pv (flinke groei verwacht) en de effecten en mogelijkheden van zonneparken, komen we tot het volgende voorstel: \textbf{\em{in 2030 zijn ontwikkelaars van zonneparken verplicht om 5\% van de investering naar verbetering van het landschap te laten gaan.}} Deze vereiste wordt opgenomen in de Zonneladder. Het is geïnspireerd op de Percentageregeling voor Beeldende Kunst (1\% investeringsvereiste in beeldende kunst bij nieuwbouw/verbouw van overheidsgebouwen). \textbf{\em{Zo kan het risico van landschapsschade worden omgezet in verbetering van het landschap.}}

\textbf{\em{De gemeente en/of provincie kunnen deze aanvullende eis stellen.}} Een ontwikkelaar moet namelijk daar een uitgebreide Omgevingsvergunning of aanpassing/wijziging van het bestemmingsplan aanvragen, omdat de functie ‘zonnepark’ niet in een standaard bestemmingsplan staat. Voor een goede ruimtelijke inpassing is maatwerk nodig, dat de gemeente beter kan leveren van het Rijk. Dit sluit aan bij de geest van Omgevingswet, die vertrouwen stelt in participatie en lokaal oplossingsvermogen.

\textbf{\em{Omwonenden mogen beslissen aan welk(e) doel(en) ze dit bedrag willen besteden.}} Te kiezen is uit natuurlijke afscherming, educatie, versterking biodiversiteit. Als bewoners in vroegtijdig stadium worden betrokken en een zekere mate van zeggenschap hebben, zal de weerstand minder zijn. Dit is in de geest van het Klimaatakkoord, dat ook eigenaarschap aan burgers wil geven in de energietransitie. Zij krijgen in de procedure zoals die nu bestaat de mogelijkheid om bezwaar te maken – wat het proces vertraagt. Wanneer bewoners meteen worden meegenomen kan dit worden voorkomen, daarbij winnen bewoners en de ontwikkelaar van het zonnepark.
\end{aanbevelingen}


\paragraph{Literatuur}
[i] Rijksoverheid (2019). Nationaal Klimaatakkoord – elektriciteit. Link: \url{https://www.klimaatakkoord.nl/elektriciteit}
 
[ii] NOS (2020). Plannen energieregio's: liever niet meer windturbines, maar zonne-energie. Link: \url{https://nos.nl/artikel/2332918-plannen-energieregio-s-liever-niet-meer-windturbines-maar-zonne-energie.html})
 
[iii] Ploum Rotterdam Law Firm (2019). De zonneladder komt eraan: Realisatie nieuwe zonneparken aan banden gelegd . Link: https://www.ploum.nl/de-zonneladder-komt-eraan-realisatie-nieuwe-zonneparken-aan-banden-gelegd/)

[iv] Rijksoverheid (2018). Nationale Woonagenda. Link: https://www.rijksoverheid.nl/documenten/publicaties/2018/05/23/nationale-woonagenda-2018-2021

[v] Binnenlands Bestuur (2020). Top 10 opvallendste feiten over zonneparken. Link: https://www.binnenlandsbestuur.nl/ruimte-en-milieu/kennispartners/kadaster/top-10-opvallendste-feiten-over-zonneparken.12606028.lynkx

[vi] Noord Holland Nieuws (2020). Grote weerstand plannen zonneparken voor de kust van Andijk. Link: https://www.nhnieuws.nl/nieuws/271852/grote-weerstand-plannen-zonneparken-voor-de-kust-van-andijk

[vii] Tweede Kamer (2018). Motie van het lid Dik-Faber c.s. over een zonneladder opstellen in samenspraak met decentrale overheden. Link: \url{https://www.tweedekamer.nl/kamerstukken/moties/detail?id=2018Z17050&did=2018D46309}

[viii] Nationaal Programma RES (2019). Factsheet Zon-pv en wind op land Analyse naar opwek van hernieuwbare energie per RES-regio. Link: https://www.regionale-energiestrategie.nl/documenten/handlerdownloadfiles.ashx?idnv=1460717
 
[ix] Binnenlands Bestuur (2020). Top 10 opvallendste feiten over zonneparken. Link: https://www.binnenlandsbestuur.nl/ruimte-en-milieu/kennispartners/kadaster/top-10-opvallendste-feiten-over-zonneparken.12606028.lynkx

[x]  Spruijt, J. (2015). Wat levert een Zonneweide per ha op? Link: https://edepot.wur.nl/336567

[xi] Binnenlands Bestuur (2020). Top 10 opvallendste feiten over zonneparken. Link: https://www.binnenlandsbestuur.nl/ruimte-en-milieu/kennispartners/kadaster/top-10-opvallendste-feiten-over-zonneparken.12606028.lynkx

[xii] Holland Solar (2019). Gedragscode Zon op Land. Link: https://hollandsolar.nl/gedragscodezonopland

[xiii] Zee, van der et al. (2019). Zonneparken – natuur en landbouw. WUR. Link: https://edepot.wur.nl/475349

[xiv] BAR-Organisatie (2019). Concept - Ruimtelijke verkenning Zonneparken in het buitengebied.

\end{voorstel}
