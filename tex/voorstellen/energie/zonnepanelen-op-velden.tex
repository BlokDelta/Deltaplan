\begin{voorstel}{Zonnepanelen op velden}
\meeschrijver{Edo van Baars}

\headerimg{img/energie/zonnepanelen-op-velden-header}
% zonnepanelen-op-velden-header
% TODO: "Tegen de bouw van dit zonnepark en meerdere parken in met name het noorden van Nederland wordt bezwaar gemaakt door bewoners die vrezen voor grootschalige aantasting van het landschap." (https://nl.wikipedia.org/wiki/Zonnepark_Lange_Runde)

\begin{samenvatting}
Zonneparken worden zorgvuldig in het landschap ingepast. Zonneparken worden een plek waar biodiversiteit op de voorgrond staat. Ontwikkelaars investeren in natuur, biodiversiteit en educatie. De regio stelt eisen aan de ontwikkeling en de burger besluit mee over de investeringen van ontwikkelaars.
\end{samenvatting}

\begin{multicols}{2}

\begin{uitdaging}
In het Nationaal Klimaatakkoord is afgesproken dat in 2030 35 TW aan elektriciteit op land worden opgewekt door wind op land en zonnepanelen op velden en grote daken [i]. Grootschalige projecten waarin zonnepanelen op velden worden gebouwd, ook wel zonneparken genoemd, zijn nu al commercieel al erg aantrekkelijk [ii]. Ook in de Regionale Energie Strategieën spelen zonneparken een belangrijke rol.

Naar verwachting zal de energieproductie van zonnepanelen toenemen van 1,5 TW in 2018 naar 8,5 TW in 2030. Niet alle daken zullen geschikt zijn voor zonnepanelen en de ruimte in bebouwd gebied is al schaars. Overigens worden zonneparken meer rendabel, naarmate ze groter worden in omvang. [iii] Het is daarom aannemelijk dat zeker een kwart van de zonnepanelen dus op veld zal komen te liggen. [iv]

Hoe plaatsen we in de toekomst zonnevelden in het landschap met respect voor de natuur en in samenspraak met de lokale bevolking?
\end{uitdaging}

\begin{overwegingen}
\paragraph{De Zonneladder}
We kunnen er niet omheen dat de klassieke zonneparken impact hebben op de natuurlijke omgeving. Een zonnepark heeft een industriële uitstraling, die niet goed past tussen natuur en groene landbouwgrond. De Zonneladder is een nieuw toetsingsinstrument dat een richtlijn geeft over waar zonnepanelen te plaatsen. Zonnepanelen worden met voorkeur op de daken en gevels geplaatst of in bebouwd gebied, zoals bij parkeerplaatsen en geluidsschermen langs de weg. Daarna pas komt de plaatsing in natuur-en landbouwgebieden. De voorkeur ligt dan bij locaties zoals waterzuiveringsinstallaties, vuilnisbelten, of bermen van wegen. [v]
\paragraph{Natuur rondom het zonnepark}
Door grootschalige ruilverkaveling medio 20e eeuw zijn kleine kavels opgebroken en natuurlijke afscheidingen – zoals houtwallen – weggehaald. Juist houtwallen bieden beschutting voor vogels, egels, vossen en insecten. Door houtwallen rondom zonnevelden verplicht te maken, gaan zonneparken bijdragen aan biodiversiteit.  Op het zonnepark komen nauwelijks mensen en vervoersmiddelen waardoor dieren vrij spel hebben.

\paragraph{Groen rondom het zonnepark}
Verschillende groene elementen van het landschap kunnen in en rondom het zonnepark worden geïncorporeerd. Dijken, watergangen, houtwallen, paden of bomenrijen zorgen ervoor dat het park meer in de omgeving opgaat. Het strakke patroon van het park wordt doorbroken. Verzekeraars vragen dat zonneparken kunnen worden afgesloten. In plaats van een hek, kan een groene heg,  houtwal of brede watergangen worden gebruikt. Een groene afscheiding zorgt voor minder zichtbaarheid van het park en geeft het landschap meer diepte, het biedt een spannender en afwisselender aanzicht dan een leeg, ruim verkaveld landschap.

\paragraph{Biodiversiteit in het park}
De bodem onder zonnepanelen kan verschalen, doordat water en zonlicht niet bij de grond kunnen.  Maar er zijn ook manieren waarop zonneparken een kans kunnen zijn voor natuur en biodiversiteit.  Door voldoende ruimte tussen de panelen te laten kan de biodiversiteit worden versterkt, bijvoorbeeld wanneer er mossen kunnen groeien, die goed met schaduw kunnen omgaan. Regenwater kan ondanks de panelen alsnog de bodem bereiken. Zo droogt de grond niet uit en krijgt het bodemleven weer een kans. [vi]

Bij nog meer ruimte tussen de panelen kunnen er schapen en kippen worden gehouden. Insectenhotels, bijenkorven en nestkasten kunnen tussen de panelen worden geplaatst. Het graven van kleine poelen tussen en naast panelen kan ook andere flora en fauna trekken. De functie van een zonnepark kan zelfs nog breder worden getrokken. Door wandelpaden door de zonnevelden aan te leggen, wordt het een plek om te recreëren. Schoolklassen kunnen zonnevelden bezoeken om te leren over al het bodemleven en duurzame energie.[vii]

\paragraph{TODO: geld voor de natuur}
[Om de 5\%-regeling te onderbouwen]

\end{overwegingen}

\begin{aanbevelingen}
We gaan ontwikkelaars van zonneparken verplicht om 5\% van de investering naar verbetering van het landschap te laten gaan. Dit wordt opgenomen in de Zonneladder. Het is geïnspireerd op de Percentageregeling voor Beeldende Kunst (1\% investeringsvereiste in beeldende kunst bij nieuwbouw/verbouw van overheidsgebouwen). De gemeente en de provincie kunnen deze aanvullende eis stellen aan de ontwikkelaar als het bestemmingsplan moet worden gewijzigd.  Zo kan er lokaal worden gekeken welke eisen precies nodig zijn.  Bewoners krijgen een stem in de besteding van de 5\%.  Dit geld kan worden geïnvesteerd in natuurlijke afscherming, educatie of versterking van biodiversiteit.  Het  risico op landschapsschade wordt omgezet in verbetering van het landschap.
\end{aanbevelingen}


\paragraph{Literatuur}
[i] Rijksoverheid (2019). Nationaal Klimaatakkoord – elektriciteit. Link: \url{https://www.klimaatakkoord.nl/elektriciteit}

[ii] Ploum Rotterdam Law Firm (2019). De zonneladder komt eraan: Realisatie nieuwe zonneparken aan banden gelegd . Link: https://www.ploum.nl/de-zonneladder-komt-eraan-realisatie-nieuwe-zonneparken-aan-banden-gelegd/)

[iii] Spruijt, J. (2015). Wat levert een Zonneweide per ha op? Link: https://edepot.wur.nl/336567

[iv] Nationaal Programma RES (2019). Factsheet Zon-pv en wind op land Analyse naar opwek van hernieuwbare energie per RES-regio. Link: https://www.regionale-energiestrategie.nl/documenten/handlerdownloadfiles.ashx?idnv=1460717

[v] Tweede Kamer (2018). Motie van het lid Dik-Faber c.s. over een zonneladder opstellen in samenspraak met decentrale overheden. Link: \url{https://www.tweedekamer.nl/kamerstukken/moties/detail?id=2018Z17050&did=2018D46309}

[vi] Zee, van der et al. (2019). Zonneparken – natuur en landbouw. WUR. Link: https://edepot.wur.nl/475349

[vii] BAR-Organisatie (2019). Concept - Ruimtelijke verkenning Zonneparken in het buitengebied.

\end{multicols}

\end{voorstel}


