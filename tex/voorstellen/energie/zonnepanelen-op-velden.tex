\begin{voorstel}{Zonnepanelen op velden}
\meeschrijver{Edo van Baars}

\headerimg[Zonnepark Lange Runde in de buurt van Emmen][\copyright{} 44BART44 \CC{} BY-SA]{img/energie/zonnepanelen-op-velden-header}
% https://commons.wikimedia.org/wiki/File:ZLRLaag.jpg
% TODO: "Tegen de bouw van dit zonnepark en meerdere parken in met name het noorden van Nederland wordt bezwaar gemaakt door bewoners die vrezen voor grootschalige aantasting van het landschap." (https://nl.wikipedia.org/wiki/Zonnepark_Lange_Runde)

\begin{samenvatting}
Zonneparken worden zorgvuldig in het landschap ingepast. Zonneparken worden een plek waar biodiversiteit op de voorgrond staat. Ontwikkelaars investeren in natuur, biodiversiteit en educatie. De regio stelt eisen aan de ontwikkeling en de burger besluit mee over de investeringen van ontwikkelaars.
\end{samenvatting}

\begin{multicols}{2}

\begin{uitdaging}
In het Klimaatakkoord is afgesproken dat in 2030 35 TW aan elektriciteit op land worden opgewekt door wind- en zonne-energie op land \parencite{noauthor_klimaatakkoord_2019}. Naar verwachting zal de energieproductie van zonnepanelen toenemen van 1,5 TW in 2018 naar 8,5 TW in 2030. Niet alle daken van gebouwen zullen geschikt zijn voor zonnepanelen en de ruimte in bebouwd gebied is al schaars. Het is daarom aannemelijk dat zeker een kwart van de zonnepanelen dus op velden zal komen te liggen \parencite{nationaal_programma_res_factsheet_2019}. Hoe plaatsen we in die zonnevelden in het landschap met respect voor de natuur en in samenspraak met de lokale bevolking?
\end{uitdaging}

\begin{overwegingen}
\paragraph{De Zonneladder}
Klassieke zonneparken hebben impact op de natuurlijke omgeving. Een zonnepark heeft een industriële uitstraling, die visueel kan botsen met natuur en groene landbouwgrond. De Zonneladder is een nieuw toetsingsinstrument dat een richtlijn geeft over waar zonnepanelen te plaatsen \parencite{dik-faber_motie_2018, sluiter_zonneladder_2019}.
\todo{wat is de status van de zonneladder? Check ook wat anderen voor 'ladder'/prioritering hebben voorgesteld}
De Zonneladder geeft de volgende prioriting, van voorkeur naar laatst keus:
\begin{enumerate}
	\item op de daken en gevels
	\item bij parkeerplaatsen en geluidsschermen langs de weg
	\item buiten gebouwde omgeving: bij waterzuiveringsinstallaties, vuilnisbelten, of bermen van wegen
	\item in natuur-en landbouwgebieden
\end{enumerate}
Als we deze Zonneladder volgen, komen er dus alleen zonnevelden in natuur- en landbouwgebieden als het nergens anders meer kan.
Het is aannemelijk dat er in ieder geval buiten de gebouwde omgeving zonnepanelen moeten komen, omdat er op daken en parkeerplaatsen waarschijnlijk niet genoeg kan worden opgewekt om aan de doelen te voldoen \parencite{spruijt_wat_2015}.

\paragraph{Houtwallen}
Door grootschalige ruilverkaveling medio 20e eeuw zijn kleine kavels opgebroken en natuurlijke afscheidingen – zoals houtwallen – weggehaald. Juist houtwallen bieden beschutting voor vogels, egels, vossen en insecten. Door houtwallen rondom zonnevelden verplicht te maken, gaan zonneparken bijdragen aan biodiversiteit. Op het zonnepark komen nauwelijks mensen en vervoersmiddelen waardoor dieren vrij spel hebben.

\paragraph{Groen rondom het zonnepark}
Verschillende groene elementen van het landschap kunnen in en rondom het zonnepark worden geïncorporeerd. Dijken, watergangen, houtwallen, paden of bomenrijen zorgen ervoor dat het park meer in de omgeving opgaat. Het strakke patroon van het park wordt doorbroken. Verzekeraars vragen nu al dat zonneparken kunnen worden afgesloten: in plaats van een hek, kan een groene heg, houtwal of brede watergangen worden gebruikt. Een groene afscheiding zorgt voor minder zichtbaarheid van het park en geeft het landschap meer diepte, het biedt een spannender en afwisselender aanzicht dan een leeg, ruim verkaveld landschap.

\paragraph{Biodiversiteit in het park}
De bodem onder zonnepanelen kan verschalen, doordat water en zonlicht niet bij de grond kunnen. Maar er zijn ook manieren waarop zonneparken een kans kunnen zijn voor natuur en biodiversiteit. Door voldoende ruimte tussen de panelen te laten kan de biodiversiteit worden versterkt, bijvoorbeeld wanneer er mossen kunnen groeien, die goed met schaduw kunnen omgaan. Regenwater kan ondanks de panelen alsnog de bodem bereiken. Zo droogt de grond niet uit en krijgt het bodemleven weer een kans \parencite{van_der_zee_zonneparken_2019}.

Bij nog meer ruimte tussen de panelen kunnen er schapen en kippen worden gehouden. Insectenhotels, bijenkorven en nestkasten kunnen tussen de panelen worden geplaatst. Het graven van kleine poelen tussen en naast panelen kan ook andere flora en fauna trekken. De functie van een zonnepark kan zelfs nog breder worden getrokken. Door wandelpaden door de zonnevelden aan te leggen, wordt het een plek om te recreëren. Schoolklassen kunnen zonnevelden bezoeken om te leren over al het bodemleven en duurzame energie \parencite{bar-organisatie_verkenning_2020}.

\todo{
Ik mis nog:
\begin{itemize}
	\item wat is maximaal potentiaal?
	\item uitwerken Zonneladder + potentiaal per stap
	\item wie wijst gebieden aan?
	\item hoe zorgen we nou precies voor draagvlak? Is de percentageregeling genoeg?
\end{itemize}
}

\end{overwegingen}

\begin{aanbevelingen}
\speerpunt{De Zonneladder is leidend}
om te bepalen welke nieuwe projecten worden goedgekeurd en prioriteit krijgen. Daarmee vullen we zo slim en effectief mogelijk de beperkte ruimte in, met zo min mogelijk impact op het landschap en de natuur. De Zonneladder wordt verankerd in nationaal en regionaal beleid, en speelt zo mee bij bijvoorbeeld het toekennen van subsidies en het wijzigen van bestemmingsplannen.

\speerpunt{Beschermen van biodiversiteit wordt een eis bij nieuwe projecten.}
Het uitgangspunt moet zijn dat zonnevelden de biodiversiteit verrijken. Dat kan met houtwallen, groenstroken, het plaatsen van insectenhotels en nestkasten, en het inbedden van een park in het bestaande landschap.

\speerpunt{We stellen een Percentageregeling voor Zonnevelden in.}
Hiermee worden ontwikkelaars van zonneparken verplicht om een percentage van de investering naar verbetering van het landschap te laten gaan. De regeling is geïnspireerd op de Percentageregeling voor Beeldende Kunst, die overheden verplicht om 1\% van de investering voor nieuwbouw of verbouw van haar gebouwen in te zetten voor beeldende kunst.
Bewoners krijgen een stem in de besteding van de gereserveerde gelden.
Zo kan er lokaal worden gekeken welke eisen precies nodig zijn.
Het kan bijvoorbeeld worden geïnvesteerd in natuurlijke afscherming, educatie of versterking van biodiversiteit.
\todo{Was er niet zo'n vergelijkbare regeling? Omgevingsfonds? Maar dan voor windmolens geloof ik.}

\end{aanbevelingen}

\end{multicols}

\end{voorstel}


