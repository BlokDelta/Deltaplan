\begin{voorstel}{Wind op land}

\headerimg[Windmolenpark in Soest vanuit de lucht][\copyright{} L-BBE \CC{} BY]{img/energie/wind-op-land-header}
% https://commons.wikimedia.org/wiki/File:Soest,_Netherlands_-_panoramio_(12).jpg

\begin{samenvatting}
We verdrievoudigen het opgestelde vermogen van wind op land tot 2030. Draagvlak en inpassing in het landschap zijn cruciaal. Een investering in energie moet daarom een investering zijn in toekomstperspectief voor de regio.
\end{samenvatting}

\begin{multicols}{2}

\begin{uitdaging}
Windenergie op land is onmisbaar in de energietransitie. In 2019 stond er voor ruim 3,5 GW aan vermogen, zo’n 2000 windmolens \parencite{cbs_statline_nodate}. Dat is nog ver onder de doelstelling van 6 GW in 2020 uit het Energieakkoord van 2013 \parencite{rijksoverheid_windenergie_2016}. Windparken op land zijn nu vaak later klaar dan gepland. We moeten de capaciteit dus sterk opschroeven. Hoe zorgen we ervoor dat dat gaat lukken? En hoe passen we dat in in het landschap, en met medewerking van burgers?
\end{uitdaging}

\begin{overwegingen}
\paragraph{Windmolens op land}
Wanneer we het hebben over wind op land gaat het over zowel kleine windmolens die enkele huishoudens van elektriciteit voorzien, als de meer gangbare windturbines van zo'n 3 MW. De innovatie en efficiëntie in de windsector maakt echter dat turbines steeds groter, hoger en efficiënter worden en dus meer opwekken. Daarom wordt er tegenwoordig gerekend met grote windturbines van 5 tot 6 MW per stuk. De keuze voor het type turbine is afhankelijk van de lokale omstandigheden (ruimte, netbelasting, draagvlak, etc.). Wanneer meerdere (vaak grote) windturbines bij elkaar staan spreken we van een windpark.

\paragraph{Planning en locaties}
Op dit moment worden op tientallen plekken in Nederland windmolen(parken) ontwikkeld, vaak op plekken die zijn aangewezen als kansrijke gebieden door overheden of worden aangedragen door lokale initiatiefnemers. Tot 2030 wordt er gewerkt aan Regionale Energiestrategieën; in 30 regio’s worden plannen gemaakt voor de opwekking van wind- en zonne-energie. Hierin worden gebieden aangewezen die kansrijk zijn voor windenergie. De aanleg van windparken op land is gecompliceerd en duurt daarom lang. Een initiatiefnemer gaat door verschillende fases: een verkennings-, vergunnings- en realisatiefase. Die kunnen bij elkaar wel negen jaar duren \parencite{rijksdienst_voor_ondernemend_nederland_windenergie_nodate}.

Of een locatie kansrijk is hangt af van verschillende zaken. Allereerst van de andere al aanwezige functies. Zo is men over het algemeen terughoudend met het ontwikkelen in natuurgebieden of bij archeologische vindplaatsen en zijn er beperkingen in de buurt van bebouwing, vanwege het geluid, de slagschaduw en het fysieke risico. Daarnaast wordt er gekeken naar de afstand tot en de capaciteit van het elektriciteitsnet om te garanderen dat het project technisch en financieel haalbaar is. Ook wordt er gelet op de inpassing in het landschap, en de opstelling van de windmolens. Zo kunnen locaties langs snelwegen of bedrijventerreinen of in bepaalde typen landschappen de voorkeur hebben boven andere locaties.

\todo{
\paragraph{Het initiatief}
Anders dan bij windmolens op zee, ligt het initiatief voor een windmolenproject [meestal] niet bij de overheid.
}

\paragraph{Kosten en financiering}
De ontwikkeling van een windpark kost niet alleen tijd, maar vraagt ook om vele onderzoeken, ruimtelijke procedures, overleg met omwonenden en andere belanghebbenden. Dan zijn er nog de bouwkosten van het windpark, de aanschaf van de windturbines, de netinpassing en de kosten van beheer en onderhoud als het windpark eenmaal draait. Er zijn zodoende subsidies beschikbaar (SDE+) om windmolenparken te ontwikkelen. Er wordt echter ook aan windmolens verdiend. Zo krijgen grondeigenaren vaak een vergoeding voor het gebruik van hun grond en verdienen de initiatiefnemers aan de exploitatie. Een stelregel is dat grootschalige opwek kostenefficiënter is dan kleinschalige opwek.


\paragraph{Draagvlak en participatie}
Het draagvlak staat bij windenergie vaak onder druk; mensen voelen zich niet gehoord of vinden dat initiatieven van bovenaf worden opgelegd. Zonder draagvlak worden processen langzaam of stranden ze. Participatie in zowel het proces, als in de exploitatie is daarom een cruciale voorwaarde. Dit kan op vele manieren. Zowel via inspraak in de reguliere politieke processen als door directe participatie in het maken van de plannen. Ook financiële participatie is een bewezen middel om het draagvlak te versterken. Zo kunnen bewoners investeren in het park door aandelen te kopen of door via een energiecoöperatie (mee) te ontwikkelen. Er zijn nu al meer dan 500 coöperaties actief met 35.000 à 40.000 leden en het aantal is groeiende (RVO, 2020). Tegelijkertijd bestaat het risico dat een aanpak met vele kleinschalige initiatieven een ‘confetti’ aan windmolenparken oplevert \parencite{college_van_rijksadviseurs_via_2019}. We moeten zodoende op zoek naar een aanpak die werkt voor het draagvlak en het landschap.

\todo{
Ik mis nog:
\begin{itemize}
	\item Hoe passeren we niet de RES?
	\item Hoe werkt het herstructuren van bestaande parken?
	\item hoeveel windmolens / capaciteit is er fysiek mogelijk? (potentie)
	\item wie neemt het initiatief?
	\item zou de overheid het initiatief kunnen nemen, bv met tenderprocedures zoals bij wind op zee?
	\item hoe dan we dat draagvlak verkrijgen precies?
	\item hoe voorkomen we nu echt de hagelslag? Moet de nationale overheid niet ingrijpen?
	\item concrete maatregelen (nu te veel abstract beleidsstuk)
\end{itemize}
}
\end{overwegingen}

\begin{aanbevelingen-start}
\speerpunt{We verdrievoudigen het huidige vermogen van windenergie op land.}
Om tot 2030 de weg te bereiden naar een energieneutrale samenleving moet de opwek van windenergie op land fors groeien. Volgens de scenario’s van het PBL en Urgenda moeten we uitgaan van zeker een verdrievoudiging van het huidige vermogen, naar zo’n 11 GW. We willen de verdrievoudiging realiseren door de bouw van windmolenparken met de modernste windmolens (op dit moment gaan we uit van zo’n 5,6 MW) en het herstructureren van bestaande parken met verouderde molens.

\speerpunt{We clusteren wind op land op passende locaties.}
Om zo het landschap zo veel als mogelijk te sparen en daarmee het draagvlak voor duurzame energie te vergroten. We ontkomen er niet aan ook in het open landschap windparken te ontwikkelen, bij voorkeur op locaties waar het veel en hard waait en in grootschalige polderlandschappen. Zo voorkomen we een hagelslag aan kleine windparken en behouden we de (economische)schaalvoordelen van clustering. Draagvlak voor een dergelijke ontwikkeling is essentieel.

\speerpunt{We brengen de baten en lasten van windenergie bij elkaar.}
Tegenover een grote bijdrage aan verduurzaming van een regio moet een vorm van (financiële) compensatie staan. Afspraken over grootschalige opwek werken zo twee kanten op. We trekken hierbij lessen uit de Europese Green Deal, waar er via een ‘Just Transition Mechanism’ extra investeringen gaan naar regio’s die harder worden geraakt door de transformatie naar een groene economie. Ook moeten partijen uit de regio zoals burgers, ondernemingen en overheden verplicht kunnen deelnemen in de investeringen, en dus ook het rendement.
\end{aanbevelingen-start}

\begin{aanbevelingen-end}
\speerpunt{We zoeken de verbinding met andere opgaven in de regio.}
Verduurzaming van regio’s zou niet alleen moeten bijdragen aan het verminderen van de \COO-uitstoot, maar bijdragen aan versterken van een toekomstperspectief. Juist in de landelijke regio’s van Nederland waar veel ruimte is voor duurzame opwek spelen andere uitdagingen, zoals bevolkingskrimp of economische stagnatie. Investeren we alleen in wind en zon, maar niet in de mensen, dan roept dat weerstand op. Denk aan de exploitatie van gas in Groningen of de grote windparken in Drenthe. Mensen moeten voelen dat investeringen in verduurzaming naast een verandering van het landschap ook bijdragen aan een betere toekomst. Als verduurzaming ook een investering is in betere mobiliteit of meer regionale banen komt dat het draagvlak ten goede.

\todo{
\speerpunt{Overheden doen het vooronderzoek, zodat kleinere partijen makkelijker mee kunnen doen.}
[Geinspireerd op de tenders voor Wind op Zee. Hoe juridisch haalbaar is dit?]
}
\end{aanbevelingen-end}

\end{multicols}

\end{voorstel}
