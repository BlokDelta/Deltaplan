\begin{voorstel}{Wind op land}
\meeschrijver{Bram van den Groenendaal}

\begin{samenvatting}
De groei van windenergie op land is cruciaal voor de energietransitie. We verdriedubbelen daarom het opgestelde vermogen tot 2030. Draagvlak en inpassing in het landschap zijn cruciaal. Een investering in energie moet daarom een investering zijn in toekomstperspectief voor de regio.
\end{samenvatting}

\begin{uitdaging}
Windenergie op land is op dit moment een van de meest efficiënte bronnen van duurzame energie en daarmee onmisbaar in de energietransitie. In 2019 stond er voor ruim 3,5 GW aan vermogen, zo’n 2000 windmolens (CBS). Dat is nog ver onder de doelstelling van 6 GW in 2020 uit het Energieakkoord van 2013 (Rijksoverheid). Windparken op land zijn vaak later klaar dan gepland. Na 2020 groeit de capaciteit van de nu geplande windparken naar ca. 7,2 GW (Monitor Wind op Land 2018). Richting 2030 en 2050 is nog veel extra vermogen nodig. Hoeveel precies hangt af van de energiemix die wordt gekozen in de Regionale Energiestrategieen die nu worden opgesteld. Berekeningen van het PBL en Urgenda gaan in de meest vergaande scenario’s uit van zo’n 11 GW aan windenergie op land. Hoe dan ook zullen er op weg naar een duurzame energievoorziening op land nog honderden grote windmolens bijgebouwd moeten worden. De vraag is hoe we dit op een gedragen en eerlijke manier voor elkaar krijgen.
\end{uitdaging}

\begin{overwegingen}
\paragraph{Windmolens op land}
Wanneer we het hebben over wind op land gaat het over zowel kleine windmolens die enkele huishoudens van elektriciteit voorzien, als de meer gangbare en bekende windturbines van 3,6 MW. De innovatie en efficiëntie in de windsector maakt echter dat turbines steeds groter worden - en dus meer opwekken - zodoende wordt er tegenwoordig ook gerekend met grote windturbines van zo’n 5,6 MW per stuk. De keuze voor het type turbine is afhankelijk van de lokale omstandigheden (ruimte, netbelasting, draagvlak, etc.).  Wanneer meerdere (vaak grote) windturbines bij elkaar staan spreken we van een windpark.

\paragraph{Planning en locaties}
Op dit moment worden op tientallen plekken in Nederland windmolen(parken) ontwikkeld, vaak op plekken die zijn aangewezen als kansrijke gebieden door overheden of worden aangedragen door lokale initiatiefnemers. Tot 2030 wordt er gewerkt aan Regionale Energiestrategieën; in 30 regio’s worden plannen gemaakt voor de opwekking van wind- en zonne-energie. Hierin worden gebieden aangewezen die kansrijk zijn voor windenergie.

Of een locatie kansrijk is hangt af van verschillende zaken. Allereerst van de andere al aanwezige functies. Zo is men over het algemeen terughoudend met het ontwikkelen in natuurgebieden of bij archeologische vindplaatsen en zijn er beperkingen in de buurt van bebouwing (risico- en geluidscontouren). Daarnaast wordt er gekeken naar de afstand tot en de capaciteit van het elektriciteitsnet - is het technisch en financieel haalbaar. Ook wordt er gelet op de landschappelijke inpassing en opstelling van de windmolens. Zo kunnen locaties langs snelwegen of bedrijventerreinen of in bepaalde typen landschappen de voorkeur hebben boven andere locaties. Door deze beperkingen heeft niet elke gemeente of regio evenveel mogelijkheden voor het opwekken van windenergie. Deze grote regionale verschillen zijn een uitdaging wat betreft het draagvlak.\

\paragraph{Aanleg}
De aanleg van windparken op land is gecompliceerd en duurt daarom lang. Een initiatiefnemer gaat door verschillende fases - een verkennings- vergunnings en realisatiefase - die bij elkaar wel negen jaar kunnen duren (RVO, 2020). Tijdens deze fases wordt er goed gekeken naar de inpassing in de leefomgeving en de gevolgen voor het milieu en de directe omgeving (geluid, lucht, natuur en ecologie, slagschaduw, etc.). In een proces met allerhande stakeholders zoals bewoners, ondernemers, overheden (vaak provincie en gemeente) en netbeheerders wordt een plan ontwikkeld en doorlopen. Vaak roept de planning van windmolenparken lokale weerstand op, waardoor de processen lang duren. Het is zaak iedereen op tijd te betrekken en een eerlijk en zuiver proces te doorlopen met voldoende inspraak en invloed.

\paragraph{Kosten en financiering}
De ontwikkeling van een windpark kost niet alleen tijd, maar vraagt ook om vele onderzoeken, ruimtelijke procedures, overleg met omwonenden en andere belanghebbenden. Dan zijn er nog de bouwkosten van het windpark, de aanschaf van de windturbines, de netinpassing en de kosten van beheer en onderhoud als het windpark eenmaal draait. Er zijn zodoende subsidies beschikbaar (SDE+) om windmolenparken te ontwikkelen. Er wordt echter ook aan windmolens verdiend. Zo krijgen grondeigenaren vaak een vergoeding voor het gebruik van hun grond en verdienen de initiatiefnemers aan de exploitatie. Een stelregel is dat grootschalige opwek kostenefficiënter is dan kleinschalige opwek.

\paragraph{Draagvlak en participatie}
Het draagvlak staat bij windenergie vaak onder druk; mensen voelen zich niet gehoord of vinden dat initiatieven van bovenaf worden opgelegd. Zonder draagvlak worden processen langzaam of stranden ze. Participatie in zowel het proces, als in de exploitatie is daarom een cruciale voorwaarde. Dit kan op vele manieren. Zowel via inspraak in de reguliere politieke processen als door directe participatie in het maken van de plannen. Ook financiële participatie is een bewezen middel om het draagvlak te versterken. Zo kunnen bewoners investeren in het park door aandelen te kopen of door via een energiecoöperatie (mee) te ontwikkelen. Er zijn nu al meer dan 500 coöperaties met 35.000 à 40.000 leden actief en het aantal is groeiende (RVO, 2020). Tegelijkertijd bestaat het risico dat een aanpak met vele kleinschalige initiatieven een ‘confetti’ aan windmolenparken oplevert (CRa). We moeten zodoende op zoek naar een aanpak die werkt voor het draagvlak en het landschap.
\end{overwegingen}

\begin{aanbevelingen}
\speerpunt{We verdriedubbelen het huidige vermogen van windenergie op land.}
Om tot 2030 de weg te bereiden naar een energieneutrale samenleving moet de opwek van windenergie op land fors groeien. Volgens de scenario’s van het PBL en Urgenda moeten we uitgaan van zeker een verdriedubbeling van het huidige vermogen, naar zo’n 11 GW. We willen de verdriedubbeling realiseren door de bouw van windmolenparken met de modernste windmolens (op dit moment gaan we uit van zo’n 5,6 MW) en het herstructureren van bestaande parken met verouderde molens.

\speerpunt{We clusteren wind op land op passende locaties.}
Om zo het landschap zo veel als mogelijk te sparen en daarmee het draagvlak voor duurzame energie te vergroten. Geschikte locaties voor grootschalige opwek zijn bijvoorbeeld locaties langs infrastructuur en op en om havens en bedrijventerreinen. We ontkomen er niet aan ook in het open landschap windparken te ontwikkelen. Kijk hierbij naar locaties waar het veel en hard waait en naar grootschaliger polderlandschappen. Zo voorkomen we een hagelslag aan kleine windparken en behouden we de (economische)schaalvoordelen van clustering. Draagvlak voor een dergelijke ontwikkeling is randvoorwaardelijk.

\speerpunt{We brengen de baten en lasten van windenergie bij elkaar.}
Om het draagvlak te vergroten moet naast het zuur moet ook zoet verdeeld worden. Tegenover een grote bijdrage aan verduurzaming van een regio moet een vorm van (financiële) compensatie staan. Afspraken over grootschalige opwek werken zo twee kanten op. We trekken hierbij lessen uit de Europese Green Deal, waar er via een ‘Just Transition Mechanism’ extra investeringen gaan naar regio’s die harder worden geraakt door de transformatie naar een groene economie. Ook moeten partijen uit de regio zoals burgers, ondernemingen en overheden verplicht kunnen deelnemen in de investeringen, en dus ook het rendement.

\speerpunt{We zoeken de verbinding met andere opgaven in de regio.}
Verduurzaming van regio’s zou niet alleen moeten bijdragen aan het verminderen van de \COO uitstoot, maar bijdragen aan versterken van een toekomstperspectief. Juist in de landelijke regio’s van Nederland waar veel ruimte is voor duurzame opwek spelen andere uitdagingen, zoals bevolkingskrimp of economische stagnatie. Investeren we alleen in wind en zon, maar niet in de mensen, dan roept dat weerstand op. Denk aan de exploitatie van gas in Groningen of de grote windparken in Drenthe. Mensen moeten voelen dat investeringen in verduurzaming naast een verandering van het landschap ook bijdragen aan een betere toekomst. Als verduurzaming ook een investering is in betere mobiliteit of meer regionale banen komt dat het draagvlak ten goede. Een regionale deal over investeringen helpt zo verduurzamen en Nederland sociaal en economisch ontwikkelen.
\end{aanbevelingen}

\paragraph{Literatuur}
Centraal Bureau voor de Statistiek (2019). Windenergie op land: productie en capaciteit per provincie. \url{https://opendata.cbs.nl/statline/#/CBS/nl/dataset/70960ned/table?fromstatweb}

College van Rijksadviseurs (2019). Via Parijs, een ontwerpverkenning. https://www.collegevanrijksadviseurs.nl/adviezen-publicaties/publicatie/2019/10/17/via-parijs

Posad Spatial Strategies et. al., (2018). Ruimtelijke verkenning klimaat en energie. https://www.klimaatakkoord.nl/documenten/publicaties/2018/02/21/ruimtelijke-verkenning-energie-en-klimaat

Rijksdienst voor Ondernemend Nederland (2020). Windenergie op land. https://www.rvo.nl/onderwerpen/duurzaam-ondernemen/duurzame-energie-opwekken/windenergie-op-land

Rijksoverheid (2020) Windenergie op land. https://www.rijksoverheid.nl/onderwerpen/duurzame-energie/windenergie-op-land

\end{voorstel}
