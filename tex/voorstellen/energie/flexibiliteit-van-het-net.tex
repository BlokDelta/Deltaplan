\begin{voorstel}{Flexibiliteit van het net}
\meeschrijver{Otto Barten}

\headerimg[Elektrische deelauto's bij een laadpaal in Amsterdam][\copyright{} Brbbl \CC{} BY-SA]{img/energie/flexibiliteit-van-het-net-header}
% https://commons.wikimedia.org/wiki/File:Car2Go_Amsterdam_Smart_ED_Herengracht.JPG

\begin{samenvatting}
Met een eerlijke energieprijs die belast wordt op basis van de uitstoot, stimuleren we de ontwikkeling van een flexibel net om ons elektriciteitsverbruik zo duurzaam mogelijk te maken en de piekbelasting te verlagen
\end{samenvatting}

\begin{multicols*}{2}
\raggedcolumns

\begin{uitdaging}
Over het algemeen wordt elektriciteit pas opgewekt als er vraag naar is: fossiele centrales worden op- en af geregeld om genoeg stroom te leveren. Als windmolens en zonnepanelen onze elektriciteit opwekken is het aanbod variabel en kan deze maar beperkt worden bijgestuurd. We moeten dus elektriciteit gebruiken als het waait en als de zon schijnt. Daarnaast neemt de piekbelasting op ons elektriciteitsnet toe, doordat industriële processen worden geëlektrificeerd en we steeds meer elektrische auto’s hebben. Zo zorgen alleen al de naar verwachting 2,8 miljoen elektrische auto’s in 2030 ervoor dat de piekvraag met bijna 50\% groeit, als het verbruik van elektriciteit niet verspreid wordt over de dag \parencite{enpuls_slim_2019}. Hoe zorgen we ervoor dat de vraag naar energie het aanbod van duurzame energie gaat volgen? En hoe vlakken we de piekbelasting af?
\end{uitdaging}

\begin{overwegingen}
% Deze flexibilisering gebeurt nu al in pilots\footnote{https://vandebron.nl/elektrisch-rijden/slim-laden}\footnote{https://www.tennet.eu/nl/nieuws/nieuws/tennet-na-succesvolle-pilots-door-met-blockchain/}\footnote{https://www.elaad.nl/projects/slim-laden-in-de-praktijk/}\footnote{https://www.deltares.nl/nl/nieuws/project-slim-malen-komt-op-stoom/} en op kleine schaal\footnote{https://www.jedlix.com/nl/}\footnote{https://www.laadje.nl/ (dit product is ontwikkeld door het bedrijf van de auteur)}, maar dit moet de standaard worden. De overheid kan hier met name voor zorgen door de prikkels in het systeem goed te zetten. Op dit moment zijn de prikkels voor het flexibel gebruiken van elektriciteit namelijk vaak nog te gering, waardoor minder \COO uitstoten of de netten ontlasten niet altijd beloond wordt.

\paragraph{De flexibele consument}
De meeste huishoudens hebben een energiecontract met één vaste prijs of hooguit dag- en nachttarief, terwijl de prijs van elektriciteit voor energieleveranciers zelf wel varieert. Door zo'n contract bestaat er voor de consument geen prikkel hebben om stroom te verbruiken wanneer er duurzame opwek én overvloed is. Er is een EU-richtlijn die voorschrijft dat elke energieleverancier de optie moet aanbieden aan consumenten om variabele prijzen te betalen \parencite{europees_parlement_notitle_nodate}. Deze prijzen zijn lager als er veel duurzame opwek is. Deze richtlijn wordt nu echter niet door Nederland gehandhaafd. Met een flexibel contract kunnen huishoudens bijvoorbeeld op een ander tijdstip de wasmachine aanzetten. De elektrische auto start dan niet met opladen als de stekker in de auto gaat, maar pas als er veel duurzame opwek is. Als er te weinig duurzame opwek is, kunnen de opgeladen batterijen van elektrische auto’s zelfs terugleveren aan het net. Burgers kiezen zelf welke slimme oplossingen ze gebruiken.

\begin{infobox}{Wat is een slim elektriciteitsnet?}
Een slim elektriciteitsnet of \term{smart grid} is een netwerk met apparaten die automatisch aangaan wanneer er veel duurzame energie is. Zo houdt het net zichzelf in balans en gebruiken we zo veel mogelijk duurzame energie.
\end{infobox}

\paragraph{Flexibiliteit in het bedrijfsleven}
In 2017 verbruikte het bedrijfsleven 69\% van de totale elektriciteit \parencite{centraal_bureau_voor_de_statistiek_statline_nodate}. Door de overgang van fossiele brandstof naar duurzame elektriciteit, zal het elektriciteitsverbruik van de industrie alleen maar groeien. Het bedrijfsleven zal dus de processen zo moeten inrichten, dat ze flexibel gebruik maken van de beschikbare duurzame elektriciteit. Bijvoorbeeld kassen, koelings- en verwarmingsapparaten en gemalen gaan elektriciteit op een flexibele manier afnemen. Bedrijven besluiten zelf welk verbruik verschoven wordt: misschien kan dit wel bij een tuinder, maar niet bij een telecomaanbieder.

\begin{infobox}{Seizoensflexibiliteit}
Flexibiliteit kan niet het veranderende aanbod van wind en zon over de seizoenen heen opvangen. Ook opslag van energie in bijvoorbeeld waterstof of accu’s is hiervoor duur. Wat handiger is, is om de hoeveelheid zonnepanelen, die vooral ‘s zomers produceren, en windparken, die vooral ‘s winters stroom leveren, slim op elkaar af te stemmen. Zo hebben we waarschijnlijk maar weinig opslag nodig!
\end{infobox}

\paragraph{Prijsprikkels}
De flexibiliteit van het net berust op een marktprincipe: de vraag reageert op de prijs. Het is dan essentiëel dat de prijs eerlijk is en duurzame energie aantrekkelijk maakt. Dat is nu nog niet zo. De prijs voor elektriciteit is nu niet afhankelijk van de \COO-uitstoot: grijze stroom kost ongeveer evenveel als groene stroom. Daarnaast is de energieprijs voor consumenten nu veel hoger dan die voor grootgebruikers, omdat grootgebruikers veel minder energiebelasting betalen!

Om flexibel verbruik voor alle partijen te stimuleren, moet de beprijzing daarom eerlijk en op basis van de uitstoot zijn. Hierdoor ontstaat een situatie waarbij stroom fors duurder wordt op momenten dat er onvoldoende duurzame energie wordt opgewekt. Maar gemiddeld hoeven we niet meer te gaan betalen: de stroomprijzen dalen juist sterk als er voldoende wind en zon is. Hierdoor wordt zowel de burger als het bedrijfsleven in gelijke mate gestimuleerd om te verbruiken op tijden dat het duurzame aanbod groot is.

\paragraph{Flexibele inzet van het net}
Deze flexibiliteit zorgt ervoor dat in de zomer ‘s nachts het overschot aan zonne-energie wordt gebruikt. Een dag met minder zon of wind wordt overbrugd met bijvoorbeeld de batterijen van onze elektrische auto’s, zonder dat er centrales aangezet worden. Zo hebben we aardgas-, biomassa- of waterstofcentrales enkel nodig als er onvoldoende zonne- en windenergie is. Flexibiliteit zorgt er ook voor dat de benodigde netinvesteringen relatief laag blijven en reduceert de behoefte aan piekvermogen.

\todo{Plaatje van een een dag opwek + verbruik. Plus: wat is de schaal van fluctuatie? Vergelijk Energy Without the Hot Air}

\todo{Zie ook: ideale energiemix, CO2 belasting}

\end{overwegingen}

\begin{aanbevelingen}
\speerpunt{Verplicht stroomaanbieders om al hun klanten de optie te geven voor flexibele tarieven.}
Stimuleer dat zowel consumenten als bedrijven hier gebruik van maken, en dat ze een zo groot mogelijk deel van hun verbruik flexibel maken. Belangrijk is dat er genoeg energieslurpers zoals elektrische auto's flexibel kunnen worden ingezet: technische innovatie hiervoor moet gestimuleerd worden. Het is daarbij essentieel dat er goede open standaarden zijn, zodat nieuwe initiatieven snel gelanceerd kunnen worden.

\todo{apart kopje voor technische innovatie en open standaarden, en daar ook iets over in de lopende tekst zetten. Dan kunnen we daar ook wat zeggen over pilots en beschikbaarheid.}

\speerpunt{Zorg dat de energieprijs is gebaseerd op de \COO-uitstoot van de opwekking, en dat deze voor iedereen gelijk is.} Dat kan met een hogere prijzen in de Europese emissiehandel in het ETS, met een \COO-belasting, of met een energiebelasting die afhankelijk is van de \COO-uitstoot. De eerste twee opties hebben onze voorkeur, omdat die de uitstoot bij de bron belasten. Voor de emissiehandel is het essentieel dat alle uitstoot onder het ETS valt, en dat er geen gratis rechten zijn. Totdat zo is kunnen we nationaal een extra belasting heffen. Ongeacht van hoe het wordt geïmplementeerd, zorgt een uitstootafhankelijke energieprijs ervoor dat duurzaam energieverbruik en flexibiliteit wordt gestimuleerd. Zo'n hervorming is ook de kans om de energiebelastingen gelijk te trekken tussen groot- en kleinverbruikers: iedereen betaalt dan een eerlijke prijs.
\end{aanbevelingen}

\end{multicols*}

\end{voorstel}
