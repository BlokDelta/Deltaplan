\begin{voorstel}{Flexibiliteit van het net}
\meeschrijver{Otto Barten}

\begin{samenvatting}
Om klimaatneutraal te worden, moeten we de flexibiliteit van het net zo goed mogelijk benutten. Op dit moment komt dit onvoldoende uit de verf, vooral omdat de overheid de prikkels niet goed heeft gezet. Delta bepleit meer flexibele contracten en een overheid die gaat beprijzen op basis van \COO-uitstoot. Burger en bedrijfsleven worden zo gestimuleerd om zelf de flexibiliteit te ontwikkelen die ons klaarmaakt voor 100\% klimaatneutraal!
\end{samenvatting}

\begin{uitdaging}
In de klassieke stroommarkt volgt het aanbod de vraag in een vast dag/nachtpatroon. Het aanbod past zich aan doordat fossiele centrales op- en af worden geregeld. Als onze opwek grotendeels van windmolens en zonnepanelen komt, varieert het aanbod echter, en kan dit aanbod de vraag maar beperkt volgen. Deze uitdaging wordt nu opgelost door aardgas- en kolencentrales als backupvermogen te gebruiken. Deze centrales kunnen we vervangen door biomassacentrales, maar zoals elders besproken is biomassa schaars. Of we kunnen ze vervangen door waterstofcentrales, maar waterstof is erg duur en energie-inefficiënt. Handiger is het om zoveel mogelijk onze elektriciteit te gebruiken wanneer de duurzame opwek het grootst is: als het waait en de zon schijnt dus!

Een tweede uitdaging is dat naarmate er meer elektriciteit verbruikt wordt door industrie, elektrische auto’s en warmtepompen, de belasting van het net groter wordt. Het huidige elektriciteitsnet op zowel lokaal als nationaal niveau is niet ontworpen voor deze taak en zal in sommige gevallen niet toereikend zijn. Zo zorgen alleen al de 2,8 miljoen elektrische auto’s, die met het huidige beleid in 2030 verwacht worden, dat de piekvraag met bijna 50\% groeit, als de flexibiliteit van het stroomnet niet benut zou worden.\footnote{https://www.enpuls.nl/media/2554/rapport-slim-laden.pdf} En uiteindelijk moeten dat er nog zo’n vijf miljoen extra worden! Daarom moet het elektrisch verbruik zoveel mogelijk geflexibiliseerd worden. Flexibilisering betekent dat ten eerste de pieken en dalen in het verbruik verminderd worden waardoor er minder netverzwaring nodig is. 

Deze flexibilisering gebeurt nu al in pilots\footnote{https://vandebron.nl/elektrisch-rijden/slim-laden
}\footnote{https://www.tennet.eu/nl/nieuws/nieuws/tennet-na-succesvolle-pilots-door-met-blockchain/
}\footnote{https://www.elaad.nl/projects/slim-laden-in-de-praktijk/
}\footnote{https://www.deltares.nl/nl/nieuws/project-slim-malen-komt-op-stoom/} en op kleine schaal\footnote{https://www.jedlix.com/nl/}\footnote{https://www.laadje.nl/ (dit product is ontwikkeld door het bedrijf van de auteur)}, maar dit moet de standaard worden. De overheid kan hier met name voor zorgen door de prikkels in het systeem goed te zetten. Op dit moment zijn de prikkels voor het flexibel gebruiken van elektriciteit namelijk vaak nog te gering, waardoor minder \COO uitstoten of de netten ontlasten niet altijd beloond wordt. \end{uitdaging}

\begin{overwegingen}
Op dit moment is de elektriciteitsvraag grotendeels inflexibel. Hoewel de prijs van elektriciteit op de spotmarkt al varieert, hebben veel huishoudens een contract met één vaste prijs, of hooguit dag- en nachttarief. Dat betekent dat ze geen prikkel hebben om stroom te verbruiken wanneer er duurzame opwek is, en ruimte op het net.

Er is een EU richtlijn die voorschrijft dat elk energiebedrijf de optie moet aanbieden aan consumenten om variabele prijzen te betalen, die lager zijn als er veel duurzame opwek is. Zo kan de burger er voor kiezen om deel te nemen aan dit \term{smart grid} (zie kader). Deze richtlijn wordt nu echter niet door Nederland gehandhaafd. We zouden dat wel moeten doen, omdat variabel beprijzen een stap is richting energie gebruiken als er veel wind en zon is.

\begin{infobox}{}
Een smart grid of slim elektriciteitsnet is een netwerk met apparaten die zichzelf automatisch aanzetten wanneer er veel duurzame opwek, en ruimte op het net is. Zo houdt het net zichzelf in balans en gebruiken we zo veel mogelijk wind- en zonne-energie.
\end{infobox}

Flexibilisering van het elektrisch gebruik kan door een grote verscheidenheid aan partijen. Huishoudens hebben hier zelf technische mogelijkheden voor, zoals op een ander tijdstip de wasmachine of de vaatwasser aanzetten. Of, waarschijnlijk belangrijker, de elektrische auto op een ander tijdstip opladen of de warmtepomp even uitzetten als er een piek in de vraag optreedt of als er weinig duurzaam vermogen is. Elektrische auto’s kunnen zelfs gaan terugleveren op deze tijden. Dit hoeven particulieren niet zelf te doen, en ook de overheid hoeft dit niet zelf te verzinnen: bedrijven zullen met slimme oplossingen komen waarmee de burger bediend wordt. De overheid houdt slechts het overzicht en stuurt met prikkels of wetgeving bij als het belang van de burger of het klimaatdoel in het geding komt. Misschien besluiten straks de meeste mensen dat ze best hun elektrische auto ‘s nachts op kunnen laden als ze daar een vergoeding van 1 euro per keer voor krijgen, maar dat ‘s nachts de was doen voor 5 cent per keer te onhandig is. Daar is niks mis mee.

Het bedrijfsleven nam in 2017 69\% van het elektrisch verbruik voor z’n rekening\footnote{https://gef.eu/wp-content/uploads/2019/11/2019.11.20-Handvest-voor-de-Slimme-Stad-webversie.pdf p. 98}, en zal waarschijnlijk, vanwege electrificatie van een deel van de industrie, ook in de toekomst een groter aandeel houden dan de burger. Hier zit ook het grootste deel van de flexibiliteit. Voorbeelden van industriële apparaten die te verschuiven zijn in de tijd zijn koelings- en verwarmingsapparaten, e-boilers, gemalen, kassen en warmte-krachtkoppelingsinstallaties (WKK). Ook een bedrijf kan zelf besluiten of het verbruik verschoven wordt: misschien kan dit wel bij een koelbedrijf of een tuinder, maar niet bij een telecomaanbieder. Het is niet aan de overheid om deze beslissingen direct te nemen, maar wel om bedrijven aan te sporen. Dit moet de overheid in de eerste plaats doen met het invoeren van afdoende prikkels zoals \COO-heffingen, zodat er voldoende vraagsturing plaatsvindt. Naast de \COO-heffing kan de overheid er ook voor kiezen om de energiebelasting niet op basis van kWh te heffen, maar op basis van \COO. Ook dit zorgt voor veel extra flexibiliteit.

Deze flexibiliteit kan ervoor zorgen dat we in de zomer ‘s avonds en ‘s nachts het surplus van zonne-energie benutten dat overdag wordt opgewekt. Ook een dag met wat minder zon moeten we, met bijvoorbeeld de batterijen van onze elektrische auto’s, kunnen overbruggen zonder dat er centrales aangezet hoeven te worden. In de winter kunnen we één of twee windstille dagen overbruggen. Alleen als de wind in de winter langer niet thuis geeft, hebben we zo onze aardgas-, biomassa- of waterstofcentrales nog nodig.

\begin{infobox}{}
Flexibilisering van de netten kan wonderen doen op de korte termijn, van minuten tot enkele dagen. Maar op seizoensschaal kan flexibilisering weinig bijdragen. Ook opslag in bijvoorbeeld waterstof of accu’s is hiervoor duur. Wat handiger is, is om de hoeveelheid zonnepanelen, die vooral ‘s zomers produceren, en windparken, die vooral ‘s winters stroom leveren, slim op elkaar af te stemmen. Zo hebben we waarschijnlijk maar weinig opslag nodig!
\end{infobox}

Wat bij uitstek een politieke keuze is, is dat de benodigde heffingen rechtvaardig verdeeld worden. Op dit moment is dit niet het geval. Bij huishoudens liggen nu al grote prikkels: de belastingen op elektriciteit, aardgas en benzine zijn al equivalent aan honderden euro’s per ton \COO. De industrie echter kent deze prikkels bij lange na niet: de Europese koolstofheffing (ETS) ligt slechts op tientallen euro’s per ton. En zelfs deze magere prikkel wordt niet eens echt betaald: de industrie krijgt gratis rechten, en bovendien vaak een vrijstelling van de energiebelasting. En dat terwijl industriële tonnen \COO precies zo schadelijk zijn als die van de burger die zijn huis verwarmt of auto rijdt. Dit kan natuurlijk niet!

\begin{infobox}{}
Als de prikkels op nationaal niveau goed gezet worden, worden hiermee gelijk ook de pieken in de lokale netten verminderd. Op lokaal niveau kunnen er soms wel andere bottlenecks zijn, bijvoorbeeld als de hele straat zijn auto oplaadt wanneer er veel windenergie wordt opgewekt. Als deze pieken apart verminderd moeten worden, kunnen regionale netbeheerders gebruikmaken van bestaande regionale markten, zoals GOPACS, waarop slimme apparaten kunnen worden aangesloten. Ook kan bekeken worden of slimme contracten, waarbij de aansluitkosten gelijke tred houden met het maximaal gevraagde vermogen, een oplossing zijn.
\end{infobox}

Samengevat zijn er op dit moment drie grote obstakels voor een effectieve benutting van flexibiliteit:
\begin{enumerate}
	\item Veel stroomgebruikers hebben niet eens de mogelijkheid om te kiezen voor variabele beprijzing, waardoor ze hun vraag zouden kunnen gaan sturen.
	\item De \COO belasting die er nu is (EU-ETS), en die ervoor kan zorgen dat mensen vooral stroom gebruiken als deze duurzaam voorhanden is, is te laag om hiervoor effectief te zijn.
	\item De voornaamste prikkel die er wel is, de energiebelasting, is slecht ontworpen en onrechtvaardig: hij stimuleert flexibel verbruik niet, en kleinverbruikers betalen veel meer dan grootverbruikers.
\end{enumerate}

\end{overwegingen}

\begin{aanbevelingen}
Om richting 100\% duurzaam in 2030 te gaan, is het cruciaal dat de flexibiliteit van het net volledig benut wordt. Hiervoor doet Delta de volgende voorstellen:
\begin{itemize}
	\item Verplicht stroomaanbieders om al hun klanten de optie te geven om variabele prijzen te gaan betalen, en zo deel te gaan nemen aan het smart grid.
	\item Stimuleer dat zowel consumenten als bedrijven hier gebruik van maken, en dat ze een zo groot mogelijk deel van hun verbruik flexibel maken.
	\item Beloon flexibel energiegebruik door de energiebelasting te heffen op \COO-basis, niet op kWh-basis. Er kan veel flexibiliteit gegenereerd worden door de belasting op deze basis te heffen bij de energiecentrale in plaats van bij de verbruiker.
	\item Zet daar bovenop zwaar in op het EU-ETS: zorg dat ook deze  \COO-beprijzing veel hoger en universeel wordt (geen gratis rechten meer). Ook dit is cruciaal om de flexibiliteit te ontsluiten die nodig is voor 100\% duurzaam
	\item Om koolstoflekkage te voorkomen naar werelddelen met een minder ambitieuze klimaatpolitiek, moet op Europees niveau een importheffing (en eventueel exportsubsidie) op \COO-basis ingevoerd worden. De interne broeikasbelasting kan dan zo hoog worden als nodig, zonder dat er banen verdwijnen naar landen met een minder ambitieuze klimaatpolitiek. 
	\item Benut deze hervormingen om zowel groot- als kleinverbruik hetzelfde bedrag per uitgestoten ton \COO te laten betalen. Gelijke monniken, gelijke kappen. Dit zorgt voor een rechtvaardige energietransitie die aan mensen uit te leggen valt. Zorg dat de burger zich niet hoeft af te vragen waarom de energietransitie haar veel geld kost, maar de raffinaderij van Shell buiten schot blijft.
\end{itemize}

Als de overheid deze punten uitvoert, ontstaat een situatie waarbij stroom fors duurder wordt op momenten waarop er onvoldoende duurzame energie wordt opgewekt. Maar gemiddeld hoeven we niet meer te gaan betalen: de stroomprijzen dalen juist sterk als er voldoende wind en zon is. Hierdoor wordt zowel de burger als het bedrijfsleven in gelijke mate gestimuleerd om te verbruiken op tijden waarop er zoveel mogelijk duurzaam aanbod is, en zo weinig mogelijk vraag. Dit houdt de benodigde netinvesteringen relatief laag en reduceert de behoefte aan piekvermogen in de vorm van vervuilende aardgascentrales, impopulaire biomassacentrales of dure groene waterstof. Zo werkt het flexibele net optimaal mee en wordt het bereiken van 100\% duurzaam zo eenvoudig en goedkoop mogelijk!
\end{aanbevelingen}

\end{voorstel}
