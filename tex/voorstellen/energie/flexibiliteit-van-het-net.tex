\begin{voorstel}{Flexibiliteit van het net}
\meeschrijver{Otto Barten}

\begin{samenvatting}
Om klimaatneutraal te worden, moeten we de flexibiliteit van het net zo goed mogelijk benutten. Op dit moment komt dit onvoldoende uit de verf, vooral omdat de overheid de prikkels niet goed heeft gezet. Delta bepleit meer flexibele contracten en een overheid die gaat beprijzen op basis van \COO-uitstoot. Burger en bedrijfsleven worden zo gestimuleerd om zelf de flexibiliteit te ontwikkelen die ons klaarmaakt voor 100\% klimaatneutraal!
\end{samenvatting}

\begin{multicols}{2}

\begin{uitdaging}
In de klassieke stroommarkt volgt de opwek van elektriciteit de vraag: fossiele centrales worden op- en af geregeld om genoeg stroom te leveren. Als onze opwek grotendeels van windmolens en zonnepanelen komt, fluctueert het aanbod echter, en kan dit aanbod de vraag maar beperkt volgen. We moeten dus proberen om zoveel mogelijk onze elektriciteit te gebruiken wanneer de duurzame opwek het grootst is: als het waait en de zon schijnt. Dat is ook belangrijk om de piekbelasting van het net te verkleinen. Hoe zorgen we ervoor dat de vraag het aanbod gaat volgen?
\end{uitdaging}

\begin{overwegingen}
% Deze flexibilisering gebeurt nu al in pilots\footnote{https://vandebron.nl/elektrisch-rijden/slim-laden}\footnote{https://www.tennet.eu/nl/nieuws/nieuws/tennet-na-succesvolle-pilots-door-met-blockchain/}\footnote{https://www.elaad.nl/projects/slim-laden-in-de-praktijk/}\footnote{https://www.deltares.nl/nl/nieuws/project-slim-malen-komt-op-stoom/} en op kleine schaal\footnote{https://www.jedlix.com/nl/}\footnote{https://www.laadje.nl/ (dit product is ontwikkeld door het bedrijf van de auteur)}, maar dit moet de standaard worden. De overheid kan hier met name voor zorgen door de prikkels in het systeem goed te zetten. Op dit moment zijn de prikkels voor het flexibel gebruiken van elektriciteit namelijk vaak nog te gering, waardoor minder \COO uitstoten of de netten ontlasten niet altijd beloond wordt.

\paragraph{Flexible energiecontracten}
Op dit moment is de elektriciteitsvraag grotendeels inflexibel. Hoewel de prijs van elektriciteit voor energieleveranciers zelf al varieert, hebben veel huishoudens een contract met één vaste prijs of hooguit dag- en nachttarief. Dat betekent dat ze geen prikkel hebben om stroom te verbruiken wanneer er duurzame opwek is, en ruimte op het net. Er is een EU-richtlijn die voorschrijft dat elke energieleverancier de optie moet aanbieden aan consumenten om variabele prijzen te betalen \parencite{europees_parlement_notitle_nodate}. Deze prijzen zijn lager als er veel duurzame opwek is. Zo kan de burger er voor kiezen om deel te nemen aan dit slimme elektriciteitsnet. Deze richtlijn wordt nu echter niet door Nederland gehandhaafd.

\begin{infobox}{Wat is een slim elektriciteitsnet?}
Een slim elektriciteitsnet of \term{smart grid} is een netwerk met apparaten die zichzelf automatisch aanzetten wanneer er veel duurzame opwek, en ruimte op het net is. Zo houdt het net zichzelf in balans en gebruiken we zo veel mogelijk duurzame energie.
\end{infobox}

\paragraph{Huishoudens}
Huishoudens kunnen zelf het tijdstip van hun elektriciteitsvraag aanpassen, door bijvoorbeeld op een ander tijdstip de wasmachine of de vaatwasser aanzetten. Nog grotere impact maakt het om de elektrische auto op een ander tijdstip op te laden of de warmtepomp even uit te zetten als er een piek in de vraag optreedt of als er weinig duurzaam vermogen is. Elektrische auto’s kunnen zelfs gaan terugleveren op deze tijden. Burgers moeten zelf kunnen kiezen welke slimme oplossingen ze kiezen. Misschien besluiten straks de meeste mensen dat ze best hun elektrische auto ‘s nachts op kunnen laden als ze daar een vergoeding van 1 euro per keer voor krijgen, maar dat ‘s nachts de was doen voor 5 cent per keer te onhandig is. Dat is uiteraard prima.

\paragraph{Bedrijfsleven}
Het bedrijfsleven nam in 2017 69\% van het elektrisch verbruik voor z’n rekening \parencite{centraal_bureau_voor_de_statistiek_statline_nodate}. Dat aandeel zal waarschijnlijk groeien door elektrificatie van industriële processen. Hier zit ook het grootste deel van de flexibiliteit. Voorbeelden van industriële apparaten die te verschuiven zijn in de tijd zijn: koelings- en verwarmingsapparaten, e-boilers, gemalen, kassen en warmte-krachtkoppelingsinstallaties (WKK). Ook een bedrijf kan zelf besluiten of het verbruik verschoven wordt: misschien kan dit wel bij een koelbedrijf of een tuinder, maar niet bij een telecomaanbieder.

\paragraph{Prijsprikkels}
De flexibiliteit van het net berust op een marktprincipe: de vraag reageert op de prijs. Het is dan essentiëel dat de prijs ook eerlijk is en duurzame energie aantrekkelijk maakt.

Op dit moment is dat maar gedeeltelijk zo.

Het is niet aan de overheid om deze beslissingen direct te nemen, maar wel om bedrijven aan te sporen. Dit moet de overheid in de eerste plaats doen met het invoeren van afdoende prikkels zoals \COO-heffingen, zodat er voldoende vraagsturing plaatsvindt. Naast de \COO-heffing kan de overheid er ook voor kiezen om de energiebelasting niet op basis van kWh te heffen, maar op basis van \COO. Ook dit zorgt voor veel extra flexibiliteit.

Deze flexibiliteit kan ervoor zorgen dat we in de zomer ‘s avonds en ‘s nachts het surplus van zonne-energie benutten dat overdag wordt opgewekt. Ook een dag met wat minder zon moeten we, met bijvoorbeeld de batterijen van onze elektrische auto’s, kunnen overbruggen zonder dat er centrales aangezet hoeven te worden. In de winter kunnen we één of twee windstille dagen overbruggen. Alleen als de wind in de winter langer niet thuis geeft, hebben we zo onze aardgas-, biomassa- of waterstofcentrales nog nodig.

\paragraph{Grotere pieken} Een tweede uitdaging is dat een groeiend deel van ons energiegebruik in de vorm van elektriciteit is, om industrie processen worden geëlektrificeerd, en we steeds meer elektrische auto’s en warmtepompen hebben. De belasting van ons net wordt daarmee groter. Zo zorgen alleen al de 2,8 miljoen elektrische auto’s, die met het huidige beleid in 2030 verwacht worden, dat de piekvraag met bijna 50\% groeit, als de flexibiliteit van het stroomnet niet benut zou worden \parencite{enpuls_slim_2019}.


\begin{infobox}{Seizoensflexibiliteit}
Flexibilisering van de netten kan wonderen doen op de korte termijn, van minuten tot enkele dagen. Maar op seizoensschaal kan flexibilisering weinig bijdragen. Ook opslag in bijvoorbeeld waterstof of accu’s is hiervoor duur. Wat handiger is, is om de hoeveelheid zonnepanelen, die vooral ‘s zomers produceren, en windparken, die vooral ‘s winters stroom leveren, slim op elkaar af te stemmen. Zo hebben we waarschijnlijk maar weinig opslag nodig!
\end{infobox}

\begin{infobox}{Lokale congestie}
Als de prikkels op nationaal niveau goed gezet worden, worden hiermee gelijk ook de pieken in de lokale netten verminderd. Op lokaal niveau kunnen er soms wel andere bottlenecks zijn, bijvoorbeeld als de hele straat zijn auto oplaadt wanneer er veel windenergie wordt opgewekt. Als deze pieken apart verminderd moeten worden, kunnen regionale netbeheerders gebruikmaken van bestaande regionale markten, zoals GOPACS, waarop slimme apparaten kunnen worden aangesloten. Ook kan bekeken worden of slimme contracten, waarbij de aansluitkosten gelijke tred houden met het maximaal gevraagde vermogen, een oplossing zijn.
\end{infobox}

\paragraph{De prijs van stroom}
Als de overheid [deze punten] uitvoert, ontstaat een situatie waarbij stroom fors duurder wordt op momenten waarop er onvoldoende duurzame energie wordt opgewekt. Maar gemiddeld hoeven we niet meer te gaan betalen: de stroomprijzen dalen juist sterk als er voldoende wind en zon is. Hierdoor wordt zowel de burger als het bedrijfsleven in gelijke mate gestimuleerd om te verbruiken op tijden waarop er zoveel mogelijk duurzaam aanbod is, en zo weinig mogelijk vraag. Dit houdt de benodigde netinvesteringen relatief laag en reduceert de behoefte aan piekvermogen in de vorm van vervuilende aardgascentrales, impopulaire biomassacentrales of dure groene waterstof. Zo werkt het flexibele net optimaal mee en wordt het bereiken van 100\% duurzaam zo eenvoudig en goedkoop mogelijk!


\end{overwegingen}

\begin{aanbevelingen}
\speerpunt{Verplicht stroomaanbieders om al hun klanten de optie te geven voor flexibele tarieven.} om variabele prijzen te gaan betalen, en zo deel te gaan nemen aan het smart grid.

Stimuleer dat zowel consumenten als bedrijven hier gebruik van maken, en dat ze een zo groot mogelijk deel van hun verbruik flexibel maken.

- open standaarden

\speerpunt{Zorg dat de energieprijs is gebaseerd op de \COO-uitstoot van de opwekking, en dat deze voor iedereen gelijk is.}

- CO2 belasting (bij vervuiler!)

- energiebelasting op basis van CO2

- ETS

Beloon flexibel energiegebruik door de energiebelasting te heffen op \COO-basis, niet op kWh-basis. Er kan veel flexibiliteit gegenereerd worden door de belasting op deze basis te heffen bij de energiecentrale in plaats van bij de verbruiker.

Zet daar bovenop zwaar in op het EU-ETS: zorg dat ook deze \COO-beprijzing veel hoger en universeel wordt (geen gratis rechten meer). Ook dit is cruciaal om de flexibiliteit te ontsluiten die nodig is voor 100\% duurzaam

Om koolstoflekkage te voorkomen naar werelddelen met een minder ambitieuze klimaatpolitiek, moet op Europees niveau een importheffing (en eventueel exportsubsidie) op \COO-basis ingevoerd worden. De interne broeikasbelasting kan dan zo hoog worden als nodig, zonder dat er banen verdwijnen naar landen met een minder ambitieuze klimaatpolitiek. 

Benut deze hervormingen om zowel groot- als kleinverbruik hetzelfde bedrag per uitgestoten ton \COO te laten betalen. Gelijke monniken, gelijke kappen. Dit zorgt voor een rechtvaardige energietransitie die aan mensen uit te leggen valt. Zorg dat de burger zich niet hoeft af te vragen waarom de energietransitie haar veel geld kost, maar de raffinaderij van Shell buiten schot blijft.

\speerpunt{Stimuleer ontwikkeling van flexibele oplossingen.}

\speerpunt{Garandeer de privacy voor burgers.}
\end{aanbevelingen}

\end{multicols}

\end{voorstel}
