\begin{voorstel}{De rol van de RES}
\meeschrijver{Walter Wittkamp}

\begin{samenvatting}
De regionale energie- strategieën (RES) halen bij lange na niet wat nodig is om binnen de aanwijzingen van het IPCC (2018) te blijven, daarom pleiten wij ervoor burgerberaden in te stellen voor de gemeentelijke energietransitie die wel tot adequate, haalbare en rechtvaardige klimaatmaatregelen besluiten.
\end{samenvatting}

\begin{uitdaging}
In het klimaatakkoord is besloten dertig energieregio’s op te richten waar provincies, gemeenten en waterschappen overleggen (polderen) over de vergroening van de energievoorziening. De nadruk voor die ‘Regionale Energiestrategieën’ (RES) ligt op duurzame elektriciteit uit zon en wind, verwarming van gebouwen, en alle infrastructuur die daarvoor nodig is. In de RESsen is voorgesteld welke ambitie mogelijk is per regio (NRC 2020-1). Uiterlijk 1 oktober dienen alle energieregio’s het concept van hun RES in en dan moeten de gemeenten aan de slag. Participatie van inwoners heeft dan nog nauwelijks plaatsgevonden, omdat het lastig zou zijn voor inwoners om iets over de regionale schaal te zeggen. Dat is de uitdaging waar de gemeenten nu voor staan: samen met burgers komen tot uitvoering. 
Doel voor elektriciteit was om in 2030 landelijk 35 terawattuur (TWh) stroom op te wekken met zonnedaken, zonneweides en windparken op het Nederlandse vasteland. Dat is circa 30 procent van het huidige landelijke stroomverbruik. Uit de RES’en lijkt het alsof er mogelijkheden zijn voor 52 TWh (met als gevaar dat gemeenten denken dat het elders wel wordt opgelost).
Nog eens 49 TWh moet in 2030 opgewekt worden door grote windturbines op de Noordzee, dat is nog circa 40\%. De zee valt \underline{niet} onder een RES.

Voorlopige conclusies:  
\begin{itemize}
	\item De oorspronkelijke bedoeling, reageren op de klimaatverandering, lijkt buiten beeld te blijven (\textbf{de urgentie ontbreekt}). Zowel in het klimaatakkoord als in de RESsen blijkt niet echt dat men in Nederland snapt wat de klimaatverandering betekent. Dit sluit aan bij een groot onderzoek onder 80.000 mensen in 40 landen waaruit blijkt dat Nederland daarvan de klimaatcrisis het meest bagatelliseert (Sciencealert, 2020). Bij het opstellen van het klimaatakkoord was ook geen enkele wetenschappelijke expert betrokken (Trouw, 2019).
	\item \textbf{Overkoepelende visie en regie ontbreken.} Er wordt gekeken naar één aspect (duurzame energie) van een complex en vaak interlokaal vraagstuk dat per gemeente moet worden uitgevoerd. Er zijn (grote) afhankelijkheden tussen de regio’s (bijv. gebruik restwarmte industriële complexen via lastig of niet aan te leggen warmtenetten door de gemeentegrenzen heen). En zijn ook hele andere gerelateerde vraagstukken (bijvoorbeeld de afname van biodiversiteit, een vergrijzende samenleving, toenemende woningnood, verduurzaming van de landbouw / voedselindustrie, toenemende migratiestromen en verplichtingen uit bijvoorbeeld Europa, het Urgenda-arrest of het stikstof-dossier). Omdat vooral ingezet wordt op zonne-energie wordt het duurder dan gedacht (ongeveer 1 miljard euro als de 52 TWh wordt gehaald). Er zijn geen concrete doelstellingen of sancties als men niet aan de opgave voldoet. De klimaatakkoord benadering was een verdeel en doe bijna niets actie, polderen tot het water ons aan de lippen staat?.
	\item \textbf{De ambitie is veel te laag.} 30\% van het landelijk stroomverbruik is waarschijnlijk in 2030 veel hoger dan de 35 TWh die dat in 2019 was (o.a. meer elektrisch vervoer, productie van waterstofgas, meer datacentra, meer inwoners). De doelstelling van 30\% uit 2019 is gezien de adviezen die het IPCC (2018) geeft ook veel te laag (we moeten 3 tot 5 keer zoveel doen als afgesproken in Parijs om onder de 2° resp 1,5° opwarming te blijven). Ook speelt dat lang niet alle plannen gaan lukken (technische beperkingen en tegenslag, juridische procedures, gebrek aan interesse bij ondernemers, gedoe met vergunningen, te weinig vakspecialisten). Verder blijkt dat sommige regio’s niets of nauwelijks iets doen wat toch al niet in de pijplijn zat. Tenslotte is er een uitdaging voor de voorzieningszekerheid, omdat de zon niet altijd schijnt en de wind niet altijd waait, terwijl er wel altijd vraag naar elektriciteit is. Het gebruik van opslag in bijvoorbeeld waterstof is nog zeer inefficiënt (ESB, 2020). Tot nu toe is er alleen met heel veel overleg heel veel papier geproduceerd. Vanuit het klimaatakkoord gezien verloopt alles tot nu toe volgens plan, maar er is nog niets concreets gedaan, dat komt pas vanaf 2021. Nederland doet het al het slechtst van alle landen in Europa wat betreft duurzame energie en is één van de grootste vervuilers per hoofd van de bevolking. Met de plannen uit de RES’en raakt Nederland nog verder achterop.
	\item Draagvlak voor de RES-beslissingen van provincies en gemeenten is minimaal, zeker bij de inwoners. Er was in het RES-proces weinig invloed van de volksvertegenwoordigers in de gemeenteraden (Prins \& van de Belt, 2020). De inwoners, die dit moeten betalen en de gevolgen zullen merken, zijn tot nu toe nog veel minder betrokken geweest, de RES’en zijn vooral ideeën van bestuurders, ambtenaren en ondernemers. Er zijn wel wat uitzonderingen:
	\begin{itemize}
		\item In de gemeente Wijk bij Duurstede, waar het beleid voor zonnevelden via een open proces is opgesteld,  is een mooi voorbeeld van samenspel tussen inwoners en volksvertegenwoordigers. Dat zag er als volgt uit: een burgerpanel, samengesteld door loting met een evenredige verdeling over de drie kernen in het betreffende gebied, heeft de raad geadviseerd over het te voeren energiebeleid. Dit \href{https://zonneveldenwijkbijduurstede.nl/wp-content/uploads/2019/11/20190613-Advies-Burgerpanel-zonnevelden.pdf}{advies} is vervolgens door de raad overgenomen en vormt daarmee het kader waarbinnen voorstellen voor concrete projecten ingediend kunnen worden bij de gemeente.
		\item In Kampen vond een energietop plaats voor inwoners, bedrijven en raadsleden
		\item Er waren regionale bijeenkomsten door de RES Drechtsteden, waar volksvertegenwoordigers, inwoners, bedrijven en stakeholders met elkaar in gesprek konden over energiescenario’s.
	\end{itemize}
\end{itemize}

\end{uitdaging}

\begin{overwegingen}
Het klimaatakkoord en de aanpak via de RES’en gaat niet leiden tot een zinvolle energietransitie in Nederland, omdat de urgentie niet benoemd wordt, er geen overkoepelende visie of regie is, de ambitie te laag is en het draagvlak ontbreekt bij de inwoners en de gemeenteraden. Alles wordt weggepolderd, dadelijk krijgen we \textbf{proces geslaagd, patiënt overleden.}

Om wel tot een zinvolle (adequaat, rechtvaardig, haalbaar) energietransitie te komen is het nodig om een expertbenadering te kiezen die niet gehinderd wordt door politiek electorale of bedrijfsbelangen en wel een hoog draagvlak onder de bevolking heeft.

De Franse burgerconventie over het klimaat laat zien dat het mogelijk is een haalbaar, rechtvaardig en adequaat klimaatbeleid te ontwikkelen (NRC 2020-2). Dat niet alleen: het toont dat we in Nederland de democratie momenteel onderbenutten, dat veel kennis, creativiteit en verantwoordelijkheidsgevoel in de samenleving niet aangeboord worden.

Laten we het Franse voorbeeld (en dat van bijvoorbeeld Wijk bij Duurstede) volgen. Daarmee doorbreken we niet alleen de impasse rond klimaatbeleid maar maken we onze democratie geschikt voor de eenentwintigste eeuw.
\end{overwegingen}

\begin{aanbevelingen}
Daarom pleiten wij 
\begin{itemize}
	\item voor het instellen van \textbf{gemeentelijke burgerberaden} (zie onder andere Remkes, 2018) om voorstellen voor de \textbf{gemeentelijke energietransitie} op te stellen;
	\item voor het instellen van een \textbf{nationaal burgerberaad} om klimaatmaatregelen vast te stellen die ons beschermen tegen een onnodige verdere stijging van de temperatuur en de potentieel catastrofale gevolgen daarvan.
\end{itemize}
Een burgerforum zal wel adequate, haalbare en rechtvaardige maatregelen voorstellen. De deelnemers vertegenwoordigen immers geen politieke partij en hoeven dus geen rekening te houden met verkiezingen, gunstige media-aandacht of een achterban. Partijpolitiek en lobbygroepen hebben weinig tot geen invloed op de besluitvorming van het burgerforum, onder andere doordat de uitvoering in handen is van een onafhankelijke organisatie. En anders dan bij een referendum of enquête staat deliberatie centraal. Dat zorgt ervoor dat mensen voorbij ideologische, culturele en religieuze verschillen leren kijken en afgaan op feiten. De deelnemers krijgen tijd, informatie van experts en professionele gespreksbegeleiding, wat ze helpt in gesprek te gaan over complexe onderwerpen en tot constructieve, weldoordachte aanbevelingen te komen.
\end{aanbevelingen}

\paragraph{literatuur}

\href{https://energievoordrenthe.nl/toolkit/volksvertegenwoordigers/HandlerDownloadFiles.ashx?idnv=1688984}{Eis de regio op}: regionale democratie in de energietransitie, Annajorien Prins \& Ruben van de Belt - Beleid en Maatschappij 2020 (47) 2

https://www.sciencealert.com/how-much-do-people-around-the-world-care-about-climate-change/amp

https://www.regionale-energiestrategie.nl/default.aspx

https://zonneveldenwijkbijduurstede.nl/nieuws/advies-burgerpanel-zonnevelden-opgenomen-in-concept-beleid/

\url{https://www.nrc.nl/nieuws/2020/06/14/windmolenparken-dan-veel-liever-zonnepanelen-a4002783#/next/2020/06/15/#304} (NRC 2020-1)

\url{https://esb.nu/esb/20059835/gebruik-van-waterstof-in-elektriciteitssector-voorlopig-onnodig-en-inefficient}

\url{https://www.nrc.nl/nieuws/2020/07/03/laat-burgers-politici-helpen-organiseer-een-burgerberaad-a4004913#/handelsblad/2020/07/04/#204} (NRC 2020-2)

https://www.trouw.nl/duurzaamheid-natuur/hoe-de-wetenschap-werd-overgeslagen-bij-het-klimaatberaad~b7e00b3d/

Eindrapport Lage drempels, hoge dijken, Remkes, J, 2018 (https://www.staatscommissieparlementairstelsel.nl/documenten/rapporten/samenvattingen/12/13/eindrapport)


\end{voorstel}
