\chapter{Vertrekpunten}

Het Deltaplan is meer dan een lange lijst voorstellen. Het wordt bijeengehouden door een visie voor een rechtvaardige en duurzame samenleving en de weg daarnaartoe. Hieronder leggen we uit vanuit welk standpunt we vertrekken. Het is ons moreel en praktisch kompas in de transitie, en bij alle voorstellen houden we tegen het licht hoe goed ze erin slagen om deze vertrekpunten te realiseren. Niet elk voorstel raakt aan alle punten, maar alle voorstellen samen moeten er goed op aansluiten.

\vertrekpunt{De overheid zorgt voor de veiligheid van haar burgers}

Een overheid ontleent haar legitimiteit aan het feit dat zij zorg draagt voor het welzijn en de veiligheid van haar burgers. De overheid heeft dus een zorgplicht, en die geldt ook voor het beschermen tegen de gevaren van klimaatverandering. Dit werd recent nog bekrachtigd door de Hoge Raad in de Klimaatzaak die door Urgenda was aangespannen tegen de Nederlandse Staat. De Nederlandse overheid moet dus handelen om klimaatverandering tegen te gaan.

In nauwe relatie tot de zorgplicht staat het voorzorgsprincipe. Dit principe schrijft voorzichtigheid voor bij het handelen in situaties van wetenschappelijke onzekerheid. Wanneer er kans is dat beleid schadelijk is voor de samenleving of het milieu, moet zo worden gehandeld dat dit risico wordt ingeperkt. Dit geldt ook wanneer er nog geen wetenschappelijk waterdicht bewijs is over de aard en de grootte van dit risico. Ook al is er sprake van enige onzekerheid binnen klimaatwetenschappen over de precieze snelheid en onomkeerbaarheid van klimaatverandering, de overheid moet handelen om de potentieel grote risico’s te verkleinen.

\vertrekpunt{Er is geen tijd te verliezen}

We kunnen het ons niet veroorloven om behouden te zijn in de ambitie van de transitie, en we hoeven ook niet te proberen de ‘optimale’ transitiesnelheid te bepalen alvorens te handelen. Die optimale snelheid is namelijk: zo snel mogelijk. Bij gebruikelijke beleidsvraagstukken past het om eerst helder te krijgen wat de verschillende belangen en afwegingen zijn. Maar nu is het zo duidelijk hoe groot de gevolgen traag handelen zijn, dat gecombineerd met het risico op onomkeerbare klimaatdestabilisatie er geen tijd te verliezen is.

Zo hoeven we niet te proberen nu al de energiebehoefte en de beste energietechnologieën voor 2030 te voorspellen: alle duurzame technieken die nu voorhanden zijn, moeten we zo snel mogelijk uitrollen. Beter wedden op te veel dan te weinig paarden. We hoeven niet te wachten tot techniek bewezen en rendabel is voordat we het gaan toepassen. Dat vereist soms een sprong in het diepe en kan soms ook onvoordelig uitpakken als een technologie niet schaalbaar blijkt. Dat is nou eenmaal het gevolg van de noodzakelijke snelheid, en moet ons niet afschrikken.

We zijn behouden over grootschalige toepassingen op korte termijn van technologische ontwikkelingen zoals groene waterstof, biokunststoffen, de Hyperloop, kernfusie, staal gemaakt met waterstof. We proberen ook zonder deze nieuwe technieken een sluitend transitieplan te maken. En tegelijkertijd zijn we groot voorstander van experimenten, innovatie, en het stimuleren van technieken die nog in de steiger staan.

Er is geen maximum doel om te halen, het is geen probleem als we sneller gaan en duurzamer worden dan juridisch strikt noodzakelijk. Immers: als we overschotten aan duurzame energie hebben, of nieuwe, betere technologie waar we niet op gerekend hadden, dan kunnen we daarmee de rest van de wereld vergroenen. Dat is uiteindelijk ook de manier waarop Nederland het meeste impact kan hebben. Als klein land is onze eigen uitstoot in het wereldbeeld gezien beperkt, maar we zijn wel in staat om een sprint te trekken en zo de rest mee te krijgen.

\vertrekpunt{Echte verandering is systeemverandering}

Het is duidelijk dat we niet op de huidige voet door kunnen. De vraag is alleen hoe het anders kan. Het is niet genoeg om individuele burgers aan te spreken hun gedrag. Natuurlijk, uiteindelijk moet iedereen duurzamer eten, kleden, reizen, bouwen. En het is ook goed en belangrijk dat mensen zelf initiatief nemen. Maar het is niet voor individuen weggelegd om het hele systeem te veranderen waarin hun handelen plaatsvindt. We hebben juist politiek nodig om die systeemverandering in te zetten en om te zorgen dat iedereen mee kan bewegen in die transitie.

Politieke actie kan het gehele netwerk raken waarin vraagstukken zich afspelen. Kijk bijvoorbeeld naar een windmolen. Dat is niet alleen een turbine die energie oplevert: het is ook een onderdeel van een internationale grondstoffenketen, die begint bij ijzererts die wordt gemijnd, vervolgens wordt verscheept, wordt gesmolten tot staal dat weer in een andere fabriek wordt gebruikt om de wieken te maken. In elke stap van de keten is er een impact op het milieu, en werken er mensen in zware processen. Ook is een windmolen een object in een landschap: dat kan invloed hebben op de natuur, op de esthetiek, burgers kunnen last hebben van de slagschaduw en het geluid, of zich verbonden voelen met de molen en omarmen als onderdeel van hun omgeving. Iemand is eigenaar van de molen, heeft betaald om hem neer te zetten en er onderhoud aan te laten doen waar weer mensen voor nodig zijn die een goede opleiding moeten hebben gevolgd, en verdient er vervolgens aan.

Elk van die stappen is verbonden met sociale, technische, economische, esthetische, duurzaamheids- en ethische vraagstukken. En in elk van die stappen moet je keuzes maken. Die aspecten negeren zorgt ervoor dat de keuze wordt gemaakt door iemand met andere belangen. Een lokale energiecoöperatie kan wel een windmolen neerzetten, maar kan niet zelf garanderen dat de erts slaafvrij is gemijnd. Het mandaat voor deze keuzes ligt dus bij de politiek.

\vertrekpunt{De overheid is aan zet}

Het initiatief voor de transitie ligt bij de overheid – dat kunnen we niet aan het bedrijfsleven of aan individuele burgers overlaten. Het idee dat sectoren via ‘zelfregulering’ kunnen vergroenen en dat kritische consumenten met hun koopgedrag kunnen aansporen tot duurzame productie, is achterhaald. Het werkt simpelweg niet snel, veelomvattend en transparant genoeg. Daarom geeft de overheid sturing aan de transitie, door doelen te stellen, de weg ernaartoe te schetsen en gepaste instrumenten op stellen zoals subsidies, normen en belastingen.

Daarnaast creëert de overheid ook: ze kan risico’s op zich nemen die niet voor de private sector zijn weggelegd, en zo de wegbereider van een nieuwe markt zijn. Dat doet ze onder andere door het financieren van fundamenteel wetenschappelijk onderzoek, het investeren in nieuwe technologieën en vervolgens de afname van nog niet rendabele implementaties te garanderen, en te voorzien in infrastructuur voor de technologie.

Ten slotte bewaakt de overheid de transitie. Ze controleert of bedrijven zich aan bijvoorbeeld de extra uitstoot- en duurzaamheidsregels houden, om zo een gelijk speelveld te creëren. Daarvoor is het noodzakelijk dat de toezichthouders genoeg middelen krijgen om bedrijven weer echt goed te kunnen controleren.

\vertrekpunt{De transitie creëert gelijkheid}

De noodzakelijke hervormingen van de economie en maatschappij brengen hoge kosten met zich mee en vragen grote offers. Zo zal de consumptie worden ingeperkt, zijn luxeproducten minder vanzelfsprekend en zullen energieprijzen stijgen. Het is essentieel dat de lasten en plichten niet alleen bij de gewone burger terechtkomen. Een klimaattransitie moet dus uit meer dan een lijst van technische oplossingen bestaan. Om de transitie te laten slagen is een massale mobilisatie van motivatie, inzet en talent nodig. Dat kan alleen als iedereen zich in de plannen kan herkennen en zich er een onderdeel van voelt.

Het staat daarom voorop dat de transitie rechtvaardig moet zijn. Dat doet een beroep op solidariteit: de sterkste schouders dragen de zwaarste lasten. Dat is niet alleen eerlijk, het is ook de enige manier waarop de transitie succesvol kan zijn. Nederlanders zouden zich terecht verzetten tegen een transitie die hen onevenredig hard raakt.

Daarom moet de transitie gelijkheid bevorderen. Het is een unieke kans om de economische verhouding weer in balans te brengen. Zo kunnen we energie-armoede tegengaan: door goede isolatie zijn armere huishoudens minder geld kwijt aan de energierekening. We kunnen de belastingen eerlijker en groener maken, en de huizenvoorraad verbeteren. Het maakt de samenleving weerbaarder voor tegenslagen: de komende decennia zullen de gevolgen van klimaatverandering hard beginnen in te slaan. Als mensen bestaanszekerheid hebben, is er meer veerkracht en meer ruimte voor creativiteit. En mensen kunnen makkelijker hun baan verlaten als ze vinden dat deze geen waarde toevoegt om zo werkelijke, duurzame waarde toe te voegen aan de maatschappij.

\vertrekpunt{Burgers zijn eigenaar van de transitie}

De transitie moet meer zijn dan alleen een nationaal opgelegd project. Het is essentieel dat individuele burgers participeren en investeren, en daarvan de vruchten plukken. Dat maakt de verhouding van de winsten gelijk: het wordt zo voor iedereen mogelijk om groen kapitaal op te bouwen. Het voorkomt dat alleen grote investeerders binnenlopen op deze massale economische hervormingen, en dat de ongelijkheid alleen maar toeneemt.

Lokale instrumenten om dit te realiseren zijn bijvoorbeeld gunstige financiering van woningisolatie, het meedoen in een lokale energiecoöperatie, of gezamenlijke deelauto’s in een dorpscollectief. Grotere oplossingen zijn het bouwen van honderdduizenden betaalbare klimaatneutrale woningen, om het voor veel meer mensen mogelijk te maken om vermogen op te bouwen in hun huis. En door bijvoorbeeld alle Nederlanders eigenaar te maken van een gezamenlijk Klimaatfonds waarmee kan worden geïnvesteerd in de transitie. Het zorgt ervoor dat burgers eigenaarschap voelen over het hele plan.

\vertrekpunt{Van product naar recht}

Waar decennia neoliberaal beleid alles van waarde probeert te `verpakken' tot een verhandelbaar product (`commodificatie'), streven wij ernaar de fundamentele onderdelen van ons bestaan weer te zien als een recht. Dat betekent dat iedere Nederlander recht heeft op goed onderwijs, toegang tot zorg, een fijne woning, en openbaar vervoer. De basisbehoeften zoals eten, drinken en kleding moeten voor iedereen betaalbaar zijn. Ook een schone lucht en toegang tot sport- en recreatievoorzieningen en natuur moet vanzelfsprekend zijn. Iedereen moet kunnen beschikken over voldoende vrije tijd en ontspanning. Mensen moeten geen angst hebben voor schulden.

Dat creëert een cultuur en een samenleving waarin immateriële zaken worden gewaardeerd, en onze belangrijkste waarden niet worden uitgedrukt in geld. Er komt meer ruimte voor vrijwilligerswerk en zorg voor elkaar. Het geeft zekerheid en stabiliteit, en daarmee mentale ruimte voor weerbaarheid, creativiteit en welzijn. Het maakt het mogelijk om burgerschap te tonen, om initiatieven te ontplooien en het geeft de tijd om je te mengen in het publieke debat. Zaken die essentieel zijn in tijden van klimaatverandering.

\vertrekpunt{Waardeer publieke investeringen}

De overheid is de belangrijkste investeerder van Nederland. Zo investeert ze in infrastructuur, veiligheid, gezondheid, fundamenteel onderzoek, innovatie, de rechtsstaat, sociale zekerheid en onderwijs. Dat zijn fundamentele drijvers van onze maatschappij. Deze investeringen creëren een stabiele basis voor bedrijven: gezonde en goed geschoolde werknemers, juridische zekerheid, nieuwe technologieën en nieuwe markten waar de overheid de wegbereider van is. De private sector profiteert dus volop van de kracht van de overheid.

Tegelijkertijd geeft het bedrijfsleven te weinig terug aan de samenleving die de investeringen heeft opgebracht. Er wordt op creatieve en soms illegale manier zo weinig mogelijk belasting betaald. Bedrijven zitten vaak op of over de grens van toegestane uitstoot- en milieunormen, en proberen dat te verhullen door groene reclamecampagnes. Met patenten houden ze innovatie die in eerste instantie grotendeels door de overheid is gefinancierd achter voor het publiek. In crisistijd worden grote bedrijven en banken gered, terwijl ze daarna in betere tijden hetzelfde risicovolle gedrag vertonen waar de crisis mee begon. Kortom: de samenleving draagt kosten, maar de winst is voor het bedrijfsleven.

We moeten daarom de overheid weer gaan waarderen als investeerder en schepper van van een goed ondernemersklimaat. Dat betekent dat we erkennen dat de overheid risico mag nemen, en daarin mag falen. En dat de investering van de overheid hoog mogen zijn, omdat de winsten nog hoger zijn en worden verdeeld over de gehele samenleving. Daar staat tegenover dat het bedrijfsleven haar eerlijke deel daarvoor moet betalen met hogere belastingen. En dat ze er niet meer mee weg komt om de regels – over belastingen, concurrentie of milieu – te omzeilen. Dat creëert publieke trots over de kwaliteit en waarde van onze publieke diensten.

\vertrekpunt{Laat groeien wat van waarde is}

Een van de oorzaken van de klimaatcrisis is de manier waarop we succes definiëren en meten. Door een blinde fixatie op het Bruto Binnenlands Product (BBP) als maatstaf voor de economie, en de staat van de economie als maatstaf voor ons algehele succes, zie we niet hoe beperkt zo’n instrument is. Het BBP meet niet hoe duurzaam een land is, hoe gelukkig mensen zijn, hoe zelfstandig burgers zijn. Het BBP stijgt bij elk gasveld dat wordt gevonden en geëxploiteerd, bij elke reparatie door aardbevingsschade, en bij elk vliegtuig dat opstijgt.

We moeten daarom het BBP als instrument en het economisch denken als geheel de deur uitdoen. De staat van de economie is maar een deel van de staat van de wereld. We moeten bepalen wat van waarde is – schone lucht, zelfredzaamheid, een steeds lagere uitstoot, vrij tijd, zorg voor elkaar, vrijheid, natuur, toegankelijke musea, economische gelijkheid – en daar op aansturen. Daarmee verdwijnt de wedstrijd om economische groei, en laten we groeien wat van waarde is.

\vertrekpunt{Betaal de ware prijs}

De prijs van producten en diensten moet de werkelijke prijs reflecteren: niet alleen de kosten die de leverancier heeft gemaakt, maar ook compensatie voor milieu- en sociale schade veroorzaakt door het productieproces. Hiermee verplaats je de kosten van de schade naar de producenten en indirect de gebruiker van het product. Zo krijgt de afnemer een eerlijk beeld van de werkelijke belasting op de aarde en worden duurzame opties aantrekkelijker.

Momenteel zijn veel producten nu goedkoper dan verantwoord is. Denk aan een goedkoop vliegticket, vlees van vee gevoerd met soja waar regenwoud voor is gekapt, en een spijkerbroek waarvan de katoenproductie enorm veel water kost en rivieren vergiftigt. Die schade kan veel vormen aannemen: uitstoot van broeikasgassen, het bedreigen van de leefomgeving van diersoorten, verlies aan biodiversiteit, verontreiniging van water, lucht en bodem, sociale uitbuiting en het aanwakkeren van conflict in instabiele regio's. Doordat deze schade niet is meegerekend in de prijs die ervoor betaald wordt, worden duurzame producten benadeeld.

Er zijn verschillende instrumenten om het speelveld gelijker te maken. Zo kun je producten met een broeikasgasheffing laten betalen voor hun uitstoot met een broeikasgasheffing belasten, of de BTW vervangen door een ‘BTE’: een Belasting Toegevoegde Emissie.  Belangrijk is dat de methode voor zowel bedrijven als consumenten gelijk werkt. Zo is de energiebelasting nu veel hoger voor consumenten dan voor bedrijven, en worden bedrijven niet gedwongen om hun energieverbruik te verlagen. Een ware prijs geldt voor iedereen.

\vertrekpunt{De vervuiler betaalt}

`De vervuiler betaalt' is een drieledig principe: technisch, moreel en praktisch. Technisch of boekhoudkundig betekent het dat belasting van milieuschade of uitstoot wordt betaald door de veroorzaker ervan: een bronbelasting. Moreel zegt het dat vervuilers moeten opdraaien voor de kosten van de vervuiling, en niet de getroffen gebruikers en beheerders van bijvoorbeeld vergiftigde rivieren, dode sloten door overbemesting, of mensen die de gevolgen van klimaatverandering door een hoge \COO{}-uitstoot ondervinden. De praktische gedachte is dat de grote bedrijven die nu veel uitstoten en de biodiversiteit om zeep helpen, ook de technische en financiële middelen hebben om alternatieven te zoeken en te implementeren. Als je pas later in de keten – bijvoorbeeld bij consumenten in de supermarkt – de verantwoordelijkheid voor duurzame keuzes legt, voelen de verantwoordelijken te weinig noodzaak om hun handelen te veranderen.

\vertrekpunt{Geld dient de werkelijke en duurzame economie}

De afgelopen decennia is geld losgezongen van zijn oorspronkelijke functie: het faciliteren van transacties en de werkelijke economie. Door financialisering – waarbij financiële producten een steeds groter deel van de economische waarde vertegenwoordigen – is de economie losgekomen van de werkelijke waarde en invloed van de onderliggende productieprocessen. Winst en economische groei worden doelen op zichzelf, en geen instrumenten om werkelijke waarde te creëren. De macht van aandeelhouders zorgt voor een kortetermijnvisie, waarbij de focus op kwartaalwinsten het bedrijven onmogelijk maakt om op langere termijn duurzame beslissingen te maken. Daarnaast is de geldscheppende functie van Centrale Banken nauwelijks onderdeel van het politieke debat, terwijl de keuzes daarin niet neutraal zijn.

Het is een politieke vraag hoe je geld inzet en hoe je het financieel systeem ontwerpt. We moeten geld dus zo ‘ontwerpen’ dat het de werkelijke economie dient, en moet zo gereguleerd worden dat het duurzame beslissingen aanmoedigt. Dat is een correctie op de opgeblazen waarde en beloningen voor sectoren zoals flitshandel die geen reële waarde toevoegen, of zelfs waarde vernietigen zoals fossiele bedrijven. De functie van geld verandert daarmee van neutraal gegeven tot onderdeel van het politieke debat, in plaats van een neutraal gegeven.

\vertrekpunt{Niet beslissen óver, maar beslissen met en door}

De transitie kun je niet van hogerhand opleggen. Er zijn talloze vraagstukken die provincies, gemeentes en burgers direct raken. Die hebben impact op bijvoorbeeld onze woningen, de energievoorziening, het landschap en de toegang tot mobiliteit. Het is dus belangrijk om waar mogelijk beslissingen lokaal te maken. Uitbreiding van lokale verantwoordelijkheid moet altijd gepaard gaan met uitbreiding van lokale macht en middelen – het mag nooit een verkapte bezuinigingsoperatie zijn.

Zo laat je ook ruimte over voor lokaal initiatief. Het is essentieel dat wooncorporaties, burgerinitiatieven en energiecoöperaties ruim baan krijgen om de transitie zelf vorm te geven. Dat vergroot niet alleen de waardering voor de veranderingen, het maakt het ook nog beter en sneller om er nieuwe oplossingen komen en iedereen meewerkt. De ambities en middelen moeten centraal worden vastgesteld, de implementatie zoveel mogelijk decentraal.

\vertrekpunt{Internationale rechtvaardigheid is leidend}

Klimaatverandering en verlies van biodiversiteit leiden tot natuurrampen, voedseltekorten, droogtes en conflicten. De landen die het hardst worden getroffen door deze klimaatcrisis hebben vaak het minst bijgedragen aan het ontstaan ervan. De veroorzakers hebben daarentegen meestal minder last van de nadelige effecten van klimaatveranderingen, door een gunstiger geografische ligging en meer middelen voor klimaatadaptatie. Nederland behoort met haar historisch gezien hoge uitstoot bij de veroorzakers van de klimaatcrisis. We hebben dus een verantwoordelijkheid om internationaal rechtvaardig klimaatbeleid aan te jagen. Dat berust op drie pijlers: ten eerste schroeven we zo snel mogelijk onze huidige uitstoot terug en sporen we andere rijke landen aan om hetzelfde te doen. Vervolgens ondersteunen we anderen landen in klimaatadaptatie, om zo de gevolgen van onder andere droogtes en natuurrampen tegen te gaan. Ten slotte compenseren we voor geleden en toekomstige schade als gevolg van klimaatverandering. Hiermee toont Nederland haar solidariteit in de internationale klimaatstrijd.

