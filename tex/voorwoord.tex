\chapter{Voorwoord}

\begin{multicols}{2}

Het huidige Nederlandse klimaatbeleid is \emph{te weinig, te laat.}
Daar zullen de meeste groene en linkse mensen het wel over eens zijn.
De klimaatcrisis raakt Nederland op alle vlakken van de maatschappij, maar de respons concentreert zich slechts op relatief smalle ambities en instrumenten.
En die ambities zijn veel te klein: als we de ergste risico's op desastreuze klimaatontwrichting willen voorkomen, moeten we alles op alles zetten om onder de anderhalf graden opwarming te blijven.
Met doelen voor 2050 redden we dat niet: als het aan ons ligt moet Nederland uiterlijk in 2030 volledig klimaatneutraal zijn.

We hebben als Nederland, met onze rijkdom en kennis, een uitgesproken rol te vervullen: in Europa, en in de rest van de wereld.
Dat is niet alleen eerlijk, dat is ook effectief.
Want als we als Nederland weer voorop lopen in de klimaatstrijd, kunnen we andere landen inspireren en meekrijgen.
Door te laten zien dat er alternatieven zijn voor het huidige beleid, creëren we momentum voor een wereldwijde klimaatbeweging.
Daar zit uiteindelijk de meeste slagkracht.

Dat is de oorsprong voor ons Deltaplan: we willen laten zien dat het kán, in 2030 klimaatneutraal zijn.
En dat het ontzettend ingrijpend is, maar we er uiteindelijk een betere, eerlijkere en schonere samenleving voor terug kunnen krijgen.
We richten ons daarbij niet alleen op de technische kant van de transitie, maar benadrukken juist ook het politieke en democratisch verhaal.\footnote{Voor een uitgebreide technische onderbouwing waarom klimaatneutraal in 2030 haalbaar is, wijzen we graag op het uitstekende rapport van Urgenda: \href{https://www.urgenda.nl/visie/rapport-2030/}{\textit{Nederland 100\% duurzame energie in 2030. Het kan als je het wilt}}}
Het is namelijk essentiëel dat de transitie breed gedragen wordt.
Net zoals de Deltawerken ons beschermen tegen het gevaar van het water en we daarmee wereldleider in watertechnologie werden, moeten we ons weren tegen de gevaren van klimaatverandering.
Een nieuwe identiteit voor Nederland staat op het spel.

Dit Deltaplan had niet tot stand kunnen komen zonder de waardevolle bijdrages van onze meeschrijvers.
We zijn hen zeer dankbaar voor het enthousiaste en geïnspireerde meedenken en vormgeven.
Deze conceptversie is een eerste versie van het `levend document' dat het Deltaplan moet zijn: we blijven eraan toevoegen en bijschaven.
Voor nu presenteren we de voorstellen uit het Thema Energie.
Ter inspiratie kun je de gehele lijst met toekomstige voorstellen alvast bekijken in de inhoudsopgave.

We hopen dat dit Deltaplan een begin is voor een nieuwe groene en linkse visie, en we horen graag wat je er van vindt.
Wil je met ons meedenken of meeschrijven? Je kan ons bereiken op \href{mailto:contact@deltagroenlinks.nl}{contact@deltagroenlinks.nl}.

\end{multicols}

\begin{centering}
Met strijdbare groet,\\
\textbf{Jantine, Daan en Rens}\\
\textit{werkgroep Delta}

\end{centering}
