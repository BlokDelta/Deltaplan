\vertrekpunt{Er is geen tijd te verliezen}

We kunnen het ons niet veroorloven om behouden te zijn in de ambitie van de transitie, en we hoeven ook niet te proberen de ‘optimale’ transitiesnelheid te bepalen alvorens te handelen. Die optimale snelheid is namelijk: zo snel mogelijk. Bij gebruikelijke beleidsvraagstukken past het om eerst helder te krijgen wat de verschillende belangen en afwegingen zijn. Maar nu is het zo duidelijk hoe groot de gevolgen traag handelen zijn, dat gecombineerd met het risico op onomkeerbare klimaatdestabilisatie er geen tijd te verliezen is.

Zo hoeven we niet te proberen nu al de energiebehoefte en de beste energietechnologieën voor 2030 te voorspellen: alle duurzame technieken die nu voorhanden zijn, moeten we zo snel mogelijk uitrollen. Beter wedden op te veel dan te weinig paarden. We hoeven niet te wachten tot techniek bewezen en rendabel is voordat we het gaan toepassen. Dat vereist soms een sprong in het diepe en kan soms ook onvoordelig uitpakken als een technologie niet schaalbaar blijkt. Dat is nou eenmaal het gevolg van de noodzakelijke snelheid, en moet ons niet afschrikken.

We zijn behouden over grootschalige toepassingen op korte termijn van technologische ontwikkelingen zoals groene waterstof, biokunststoffen, de Hyperloop, kernfusie, staal gemaakt met waterstof. We proberen ook zonder deze nieuwe technieken een sluitend transitieplan te maken. En tegelijkertijd zijn we groot voorstander van experimenten, innovatie, en het stimuleren van technieken die nog in de steiger staan.

Er is geen maximum doel om te halen, het is geen probleem als we sneller gaan en duurzamer worden dan juridisch strikt noodzakelijk. Immers: als we overschotten aan duurzame energie hebben, of nieuwe, betere technologie waar we niet op gerekend hadden, dan kunnen we daarmee de rest van de wereld vergroenen. Dat is uiteindelijk ook de manier waarop Nederland het meeste impact kan hebben. Als klein land is onze eigen uitstoot in het wereldbeeld gezien beperkt, maar we zijn wel in staat om een sprint te trekken en zo de rest mee te krijgen.
