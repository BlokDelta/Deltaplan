\vertrekpunt{De vervuiler betaalt}

`De vervuiler betaalt' is een drieledig principe: technisch, moreel en praktisch. Technisch of boekhoudkundig betekent het dat belasting van milieuschade of uitstoot wordt betaald door de veroorzaker ervan: een bronbelasting. Moreel zegt het dat vervuilers moeten opdraaien voor de kosten van de vervuiling, en niet de getroffen gebruikers en beheerders van bijvoorbeeld vergiftigde rivieren, dode sloten door overbemesting, of mensen die de gevolgen van klimaatverandering door een hoge \COO{}-uitstoot ondervinden. De praktische gedachte is dat de grote bedrijven die nu veel uitstoten en de biodiversiteit om zeep helpen, ook de technische en financiële middelen hebben om alternatieven te zoeken en te implementeren. Als je pas later in de keten – bijvoorbeeld bij consumenten in de supermarkt – de verantwoordelijkheid voor duurzame keuzes legt, voelen de verantwoordelijken te weinig noodzaak om hun handelen te veranderen.
