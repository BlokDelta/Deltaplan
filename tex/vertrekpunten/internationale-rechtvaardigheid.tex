\vertrekpunt{Internationale rechtvaardigheid is leidend}

Klimaatverandering en verlies van biodiversiteit leiden tot natuurrampen, voedseltekorten, droogtes en conflicten. De landen die het hardst worden getroffen door deze klimaatcrisis hebben vaak het minst bijgedragen aan het ontstaan ervan. De veroorzakers hebben daarentegen meestal minder last van de nadelige effecten van klimaatveranderingen, door een gunstiger geografische ligging en meer middelen voor klimaatadaptatie. Nederland behoort met haar historisch gezien hoge uitstoot bij de veroorzakers van de klimaatcrisis. We hebben dus een verantwoordelijkheid om internationaal rechtvaardig klimaatbeleid aan te jagen. Dat berust op drie pijlers: ten eerste schroeven we zo snel mogelijk onze huidige uitstoot terug en sporen we andere rijke landen aan om hetzelfde te doen. Vervolgens ondersteunen we anderen landen in klimaatadaptatie, om zo de gevolgen van onder andere droogtes en natuurrampen tegen te gaan. Ten slotte compenseren we voor geleden en toekomstige schade als gevolg van klimaatverandering. Hiermee toont Nederland haar solidariteit in de internationale klimaatstrijd.
