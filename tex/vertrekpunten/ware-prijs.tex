\vertrekpunt{Betaal de ware prijs}

De prijs van producten en diensten moet de werkelijke prijs reflecteren: niet alleen de kosten die de leverancier heeft gemaakt, maar ook compensatie voor milieu- en sociale schade veroorzaakt door het productieproces. Hiermee verplaats je de kosten van de schade naar de producenten en indirect de gebruiker van het product. Zo krijgt de afnemer een eerlijk beeld van de werkelijke belasting op de aarde en worden duurzame opties aantrekkelijker.

Momenteel zijn veel producten nu goedkoper dan verantwoord is. Denk aan een goedkoop vliegticket, vlees van vee gevoerd met soja waar regenwoud voor is gekapt, en een spijkerbroek waarvan de katoenproductie enorm veel water kost en rivieren vergiftigt. Die schade kan veel vormen aannemen: uitstoot van broeikasgassen, het bedreigen van de leefomgeving van diersoorten, verlies aan biodiversiteit, verontreiniging van water, lucht en bodem, sociale uitbuiting en het aanwakkeren van conflict in instabiele regio's. Doordat deze schade niet is meegerekend in de prijs die ervoor betaald wordt, worden duurzame producten benadeeld.

Er zijn verschillende instrumenten om het speelveld gelijker te maken. Zo kun je producten met een broeikasgasheffing laten betalen voor hun uitstoot met een broeikasgasheffing belasten, of de BTW vervangen door een ‘BTE’: een Belasting Toegevoegde Emissie.  Belangrijk is dat de methode voor zowel bedrijven als consumenten gelijk werkt. Zo is de energiebelasting nu veel hoger voor consumenten dan voor bedrijven, en worden bedrijven niet gedwongen om hun energieverbruik te verlagen. Een ware prijs geldt voor iedereen.
