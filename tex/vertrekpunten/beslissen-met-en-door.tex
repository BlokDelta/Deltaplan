\vertrekpunt{Niet beslissen óver, maar beslissen met en door}

De transitie kun je niet van hogerhand opleggen. Er zijn talloze vraagstukken die provincies, gemeentes en burgers direct raken. Die hebben impact op bijvoorbeeld onze woningen, de energievoorziening, het landschap en de toegang tot mobiliteit. Het is dus belangrijk om waar mogelijk beslissingen lokaal te maken. Uitbreiding van lokale verantwoordelijkheid moet altijd gepaard gaan met uitbreiding van lokale macht en middelen – het mag nooit een verkapte bezuinigingsoperatie zijn.

Zo laat je ook ruimte over voor lokaal initiatief. Het is essentieel dat wooncorporaties, burgerinitiatieven en energiecoöperaties ruim baan krijgen om de transitie zelf vorm te geven. Dat vergroot niet alleen de waardering voor de veranderingen, het maakt het ook nog beter en sneller om er nieuwe oplossingen komen en iedereen meewerkt. De ambities en middelen moeten centraal worden vastgesteld, de implementatie zoveel mogelijk decentraal.
