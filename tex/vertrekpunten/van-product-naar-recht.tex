\vertrekpunt{Van product naar recht}

Waar decennia neoliberaal beleid alles van waarde probeert te `verpakken' tot een verhandelbaar product (`commodificatie'), streven wij ernaar de fundamentele onderdelen van ons bestaan weer te zien als een recht. Dat betekent dat iedere Nederlander recht heeft op goed onderwijs, toegang tot zorg, een fijne woning, en openbaar vervoer. De basisbehoeften zoals eten, drinken en kleding moeten voor iedereen betaalbaar zijn. Ook een schone lucht en toegang tot sport- en recreatievoorzieningen en natuur moet vanzelfsprekend zijn. Iedereen moet kunnen beschikken over voldoende vrije tijd en ontspanning. Mensen moeten geen angst hebben voor schulden.

Dat creëert een cultuur en een samenleving waarin immateriële zaken worden gewaardeerd, en onze belangrijkste waarden niet worden uitgedrukt in geld. Er komt meer ruimte voor vrijwilligerswerk en zorg voor elkaar. Het geeft zekerheid en stabiliteit, en daarmee mentale ruimte voor weerbaarheid, creativiteit en welzijn. Het maakt het mogelijk om burgerschap te tonen, om initiatieven te ontplooien en het geeft de tijd om je te mengen in het publieke debat. Zaken die essentieel zijn in tijden van klimaatverandering.
