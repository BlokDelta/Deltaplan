\vertrekpunt{Laat groeien wat van waarde is}

Een van de oorzaken van de klimaatcrisis is de manier waarop we succes definiëren en meten. Door een blinde fixatie op het Bruto Binnenlands Product (BBP) als maatstaf voor de economie, en de staat van de economie als maatstaf voor ons algehele succes, zie we niet hoe beperkt zo’n instrument is. Het BBP meet niet hoe duurzaam een land is, hoe gelukkig mensen zijn, hoe zelfstandig burgers zijn. Het BBP stijgt bij elk gasveld dat wordt gevonden en geëxploiteerd, bij elke reparatie door aardbevingsschade, en bij elk vliegtuig dat opstijgt.

We moeten daarom het BBP als instrument en het economisch denken als geheel de deur uitdoen. De staat van de economie is maar een deel van de staat van de wereld. We moeten bepalen wat van waarde is – schone lucht, zelfredzaamheid, een steeds lagere uitstoot, vrij tijd, zorg voor elkaar, vrijheid, natuur, toegankelijke musea, economische gelijkheid – en daar op aansturen. Daarmee verdwijnt de wedstrijd om economische groei, en laten we groeien wat van waarde is.
