\vertrekpunt{Waardeer publieke investeringen}

De overheid is de belangrijkste investeerder van Nederland. Zo investeert ze in infrastructuur, veiligheid, gezondheid, fundamenteel onderzoek, innovatie, de rechtsstaat, sociale zekerheid en onderwijs. Dat zijn fundamentele drijvers van onze maatschappij. Deze investeringen creëren een stabiele basis voor bedrijven: gezonde en goed geschoolde werknemers, juridische zekerheid, nieuwe technologieën en nieuwe markten waar de overheid de wegbereider van is. De private sector profiteert dus volop van de kracht van de overheid.

Tegelijkertijd geeft het bedrijfsleven te weinig terug aan de samenleving die de investeringen heeft opgebracht. Er wordt op creatieve en soms illegale manier zo weinig mogelijk belasting betaald. Bedrijven zitten vaak op of over de grens van toegestane uitstoot- en milieunormen, en proberen dat te verhullen door groene reclamecampagnes. Met patenten houden ze innovatie die in eerste instantie grotendeels door de overheid is gefinancierd achter voor het publiek. In crisistijd worden grote bedrijven en banken gered, terwijl ze daarna in betere tijden hetzelfde risicovolle gedrag vertonen waar de crisis mee begon. Kortom: de samenleving draagt kosten, maar de winst is voor het bedrijfsleven.

We moeten daarom de overheid weer gaan waarderen als investeerder en schepper van van een goed ondernemersklimaat. Dat betekent dat we erkennen dat de overheid risico mag nemen, en daarin mag falen. En dat de investering van de overheid hoog mogen zijn, omdat de winsten nog hoger zijn en worden verdeeld over de gehele samenleving. Daar staat tegenover dat het bedrijfsleven haar eerlijke deel daarvoor moet betalen met hogere belastingen. En dat ze er niet meer mee weg komt om de regels – over belastingen, concurrentie of milieu – te omzeilen. Dat creëert publieke trots over de kwaliteit en waarde van onze publieke diensten.
