\vertrekpunt{Geld dient de werkelijke en duurzame economie}

De afgelopen decennia is geld losgezongen van zijn oorspronkelijke functie: het faciliteren van transacties en de werkelijke economie. Door financialisering – waarbij financiële producten een steeds groter deel van de economische waarde vertegenwoordigen – is de economie losgekomen van de werkelijke waarde en invloed van de onderliggende productieprocessen. Winst en economische groei worden doelen op zichzelf, en geen instrumenten om werkelijke waarde te creëren. De macht van aandeelhouders zorgt voor een kortetermijnvisie, waarbij de focus op kwartaalwinsten het bedrijven onmogelijk maakt om op langere termijn duurzame beslissingen te maken. Daarnaast is de geldscheppende functie van Centrale Banken nauwelijks onderdeel van het politieke debat, terwijl de keuzes daarin niet neutraal zijn.

Het is een politieke vraag hoe je geld inzet en hoe je het financieel systeem ontwerpt. We moeten geld dus zo ‘ontwerpen’ dat het de werkelijke economie dient, en moet zo gereguleerd worden dat het duurzame beslissingen aanmoedigt. Dat is een correctie op de opgeblazen waarde en beloningen voor sectoren zoals flitshandel die geen reële waarde toevoegen, of zelfs waarde vernietigen zoals fossiele bedrijven. De functie van geld verandert daarmee van neutraal gegeven tot onderdeel van het politieke debat.
