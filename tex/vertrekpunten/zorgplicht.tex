\vertrekpunt{De overheid zorgt voor de veiligheid van haar burgers}

Een overheid ontleent haar legitimiteit aan het feit dat zij zorg draagt voor het welzijn en de veiligheid van haar burgers. De overheid heeft dus een zorgplicht, en die geldt ook voor het beschermen tegen de gevaren van klimaatverandering. Dit werd recent nog bekrachtigd door de Hoge Raad in de Klimaatzaak die door Urgenda was aangespannen tegen de Nederlandse Staat. De Nederlandse overheid moet dus handelen om klimaatverandering tegen te gaan.

In nauwe relatie tot de zorgplicht staat het voorzorgsprincipe. Dit principe schrijft voorzichtigheid voor bij het handelen in situaties van wetenschappelijke onzekerheid. Wanneer er kans is dat beleid schadelijk is voor de samenleving of het milieu, moet zo worden gehandeld dat dit risico wordt ingeperkt. Dit geldt ook wanneer er nog geen wetenschappelijk waterdicht bewijs is over de aard en de grootte van dit risico. Ook al is er sprake van enige onzekerheid binnen klimaatwetenschappen over de precieze snelheid en onomkeerbaarheid van klimaatverandering, de overheid moet handelen om de potentieel grote risico’s te verkleinen.
