\vertrekpunt{Echte verandering is systeemverandering}

Het is duidelijk dat we niet op de huidige voet door kunnen. De vraag is alleen hoe het anders kan. Het is niet genoeg om individuele burgers aan te spreken hun gedrag. Natuurlijk, uiteindelijk moet iedereen duurzamer eten, kleden, reizen, bouwen. En het is ook goed en belangrijk dat mensen zelf initiatief nemen. Maar het is niet voor individuen weggelegd om het hele systeem te veranderen waarin hun handelen plaatsvindt. We hebben juist politiek nodig om die systeemverandering in te zetten en om te zorgen dat iedereen mee kan bewegen in die transitie.

Politieke actie kan het gehele netwerk raken waarin vraagstukken zich afspelen. Kijk bijvoorbeeld naar een windmolen. Dat is niet alleen een turbine die energie oplevert: het is ook een onderdeel van een internationale grondstoffenketen, die begint bij ijzererts die wordt gemijnd, vervolgens wordt verscheept, wordt gesmolten tot staal dat weer in een andere fabriek wordt gebruikt om de wieken te maken. In elke stap van de keten is er een impact op het milieu, en werken er mensen in zware processen. Ook is een windmolen een object in een landschap: dat kan invloed hebben op de natuur, op de esthetiek, burgers kunnen last hebben van de slagschaduw en het geluid, of zich verbonden voelen met de molen en omarmen als onderdeel van hun omgeving. Iemand is eigenaar van de molen, heeft betaald om hem neer te zetten en er onderhoud aan te laten doen waar weer mensen voor nodig zijn die een goede opleiding moeten hebben gevolgd, en verdient er vervolgens aan.

Elk van die stappen is verbonden met sociale, technische, economische, esthetische, duurzaamheids- en ethische vraagstukken. En in elk van die stappen moet je keuzes maken. Die aspecten negeren zorgt ervoor dat de keuze wordt gemaakt door iemand met andere belangen. Een lokale energiecoöperatie kan wel een windmolen neerzetten, maar kan niet zelf garanderen dat de erts slaafvrij is gemijnd. Het mandaat voor deze keuzes ligt dus bij de politiek.
