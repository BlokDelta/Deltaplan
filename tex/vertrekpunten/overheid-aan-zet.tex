\vertrekpunt{De overheid is aan zet}

Het initiatief voor de transitie ligt bij de overheid – dat kunnen we niet aan het bedrijfsleven of aan individuele burgers overlaten. Het idee dat sectoren via ‘zelfregulering’ kunnen vergroenen en dat kritische consumenten met hun koopgedrag kunnen aansporen tot duurzame productie, is achterhaald. Het werkt simpelweg niet snel, veelomvattend en transparant genoeg. Daarom geeft de overheid sturing aan de transitie, door doelen te stellen, de weg ernaartoe te schetsen en gepaste instrumenten op stellen zoals subsidies, normen en belastingen.

Daarnaast creëert de overheid ook: ze kan risico’s op zich nemen die niet voor de private sector zijn weggelegd, en zo de wegbereider van een nieuwe markt zijn. Dat doet ze onder andere door het financieren van fundamenteel wetenschappelijk onderzoek, het investeren in nieuwe technologieën en vervolgens de afname van nog niet rendabele implementaties te garanderen, en te voorzien in infrastructuur voor de technologie.

Ten slotte bewaakt de overheid de transitie. Ze controleert of bedrijven zich aan bijvoorbeeld de extra uitstoot- en duurzaamheidsregels houden, om zo een gelijk speelveld te creëren. Daarvoor is het noodzakelijk dat de toezichthouders genoeg middelen krijgen om bedrijven weer echt goed te kunnen controleren.
