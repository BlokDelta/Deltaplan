\vertrekpunt{De transitie creëert gelijkheid}

De noodzakelijke hervormingen van de economie en maatschappij brengen hoge kosten met zich mee en vragen grote offers. Zo zal de consumptie worden ingeperkt, zijn luxeproducten minder vanzelfsprekend en zullen energieprijzen stijgen. Het is essentieel dat de lasten en plichten niet alleen bij de gewone burger terechtkomen. Een klimaattransitie moet dus uit meer dan een lijst van technische oplossingen bestaan. Om de transitie te laten slagen is een massale mobilisatie van motivatie, inzet en talent nodig. Dat kan alleen als iedereen zich in de plannen kan herkennen en zich er een onderdeel van voelt.

Het staat daarom voorop dat de transitie rechtvaardig moet zijn. Dat doet een beroep op solidariteit: de sterkste schouders dragen de zwaarste lasten. Dat is niet alleen eerlijk, het is ook de enige manier waarop de transitie succesvol kan zijn. Nederlanders zouden zich terecht verzetten tegen een transitie die hen onevenredig hard raakt.

Daarom moet de transitie gelijkheid bevorderen. Het is een unieke kans om de economische verhouding weer in balans te brengen. Zo kunnen we energie-armoede tegengaan: door goede isolatie zijn armere huishoudens minder geld kwijt aan de energierekening. We kunnen de belastingen eerlijker en groener maken, en de huizenvoorraad verbeteren. Het maakt de samenleving weerbaarder voor tegenslagen: de komende decennia zullen de gevolgen van klimaatverandering hard beginnen in te slaan. Als mensen bestaanszekerheid hebben, is er meer veerkracht en meer ruimte voor creativiteit. En mensen kunnen makkelijker hun baan verlaten als ze vinden dat deze geen waarde toevoegt om zo werkelijke, duurzame waarde toe te voegen aan de maatschappij.
