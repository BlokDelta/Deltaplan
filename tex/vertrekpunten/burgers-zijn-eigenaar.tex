\vertrekpunt{Burgers zijn eigenaar van de transitie}

De transitie moet meer zijn dan alleen een nationaal opgelegd project. Het is essentieel dat individuele burgers participeren en investeren, en daarvan de vruchten plukken. Dat maakt de verhouding van de winsten gelijk: het wordt zo voor iedereen mogelijk om groen kapitaal op te bouwen. Het voorkomt dat alleen grote investeerders binnenlopen op deze massale economische hervormingen, en dat de ongelijkheid alleen maar toeneemt.

Lokale instrumenten om dit te realiseren zijn bijvoorbeeld gunstige financiering van woningisolatie, het meedoen in een lokale energiecoöperatie, of gezamenlijke deelauto’s in een dorpscollectief. Grotere oplossingen zijn het bouwen van honderdduizenden betaalbare klimaatneutrale woningen, om het voor veel meer mensen mogelijk te maken om vermogen op te bouwen in hun huis. En door bijvoorbeeld alle Nederlanders eigenaar te maken van een gezamenlijk Klimaatfonds waarmee kan worden geïnvesteerd in de transitie. Het zorgt ervoor dat burgers eigenaarschap voelen over het hele plan.
