\chapter{Uitgangspunten}

\subsection{De overheid draagt zorg voor de veiligheid van burgers}

De Nederlandse overheid ontleent haar bestaansrecht aan het feit dat zij zorg draagt voor het welzijn van de Nederlandse burger. De overheid en haar burgers zijn een sociaal contract aangegaan tot bescherming. Het belangrijkste uitgangspunt voor handelen door de overheid is hierdoor de zorgplicht, het zorgen voor haar burgers. Eerder stelde de Hoge Raad in de Klimaatzaak dat door de ernst van de gevolgen van klimaatverandering en de grote kans dat – zonder maatregelen – gevaarlijke klimaatverandering zal intreden, rust op de Staat een zorgplicht om maatregelen te treffen. Niet handelen door de Staat is onrechtmatig.

In nauwe relatie tot de zorgplicht staat het voorzorgsprincipe. Dit politieke en morele principe biedt uitleg over handelen bij wetenschappelijke onderzoekerheid. Wanneer bepaald beleid mogelijk negatieve gevolgen kan hebben voor de samenleving of het milieu, moet zo worden gehandeld dat dit risico wordt ingeperkt. Ook binnen de klimaatwetenschappen is sprake van enige onzekerheid, maar het risico op grote gevaren door klimaatverandering is groot. Het voorzorgsprincipe stelt dus dat dit risico moet koste wat het kost voorkomen worden.

De zorgplicht en het voorzorgsprincipe bieden de overheid een duidelijk kader tot handelen en biedt geen ruimte voor een afwachtend beleid. Het oplossen van de klimaat- en ecologische crisis is onoverkomelijk om zo het land en haar burgers te beschermen.

\subsection{Een groene transitie moet rechtvaardig zijn, nationaal…}

De verandering richting een ecologisch duurzame samenleving kan alleen plaatsvinden als sociaaleconomische ongelijkheid wordt aangepakt. Het is duidelijk dat de klimaattransitie kosten met zich mee zal brengen. Er zullen maatregelen worden gesteld, die grenzen stellen aan economische groei en aan het winstdenken. Deze maatregelen en kosten moeten eerlijk verspreid worden over de samenleving. De basis en daarom zullen de meest welvarende mensen en bedrijven, het grootste aandeel hebben in de financiering van de oplossing. Aan de andere kant biedt een transitie met ontwikkeling van technologieën kansen en mogelijkheden tot economische vooruitgang. De overheid zal bijvoorbeeld investeren in zonnepanelen. Deze investeringen zullen niet alleen , maar juist zorgen dat iedereen een zonnepaneel zou kunnen installeren en kan profiteren van het opwekken van schone energie. 

Maatschappelijke verandering is onontkoombaar (en nodig!). Een transitie als deze, vergt een visie op maatschappij en sociale verhoudingen binnen een land. Gelijke kansen in bijvoorbeeld de zorg, de arbeidsmarkt en in het onderwijs en aan de andere kant het terugdringen van inkomensongelijkheid, zijn leidende uitgangswaarden. Een werkelijk groene en sociale transitie kan nooit ongelijkheid in de hand spelen.

\subsection{… en internationaal}
Klimaatverandering leidt tot conflicten, overstromingen en hongersnoden. Degenen die de meeste schade ondervinden van deze klimaatcrisis, hebben veel minder bijgedragen aan het ontstaan ervan. De veroorzakers daarentegen, hebben minder last van de nadelige effecten van klimaatveranderingen, bijvoorbeeld door geografische ligging en meer middelen om klimaatadaptatie te realiseren. Nederland hoort bij de veroorzakers van de klimaatcrisis, met historisch gezien een gemiddeld hoge uitstoot per capita. Internationale klimaatrechtvaardigheid berust op drie peilers: Nederland past het huidige uitstootpatroon aan om nadelige effecten van klimaatveranderingen tegen te gaan, Nederland neemt verantwoordelijkheid voor haar hoge uitstoot in het verleden en Nederland is solidariteit met landen die effecten van klimaatverandering zullen ondervinden.

\subsection{Rechtvaardig ontwikkelen}
Het uitvoeren van maatregelen en het ontwikkelen van nieuwe technologieën moet altijd getoetst worden aan een set vragen, die rechtvaardigheid uitvragen. Onder welke omstandigheden wordt een project of product ontwikkeld? Waar komen de grondstoffen vandaan? Hoe verhoudt het project of product zich tot de keten, de omgeving en de samenleving? Een product of een dienst bestaat bij de gratie van de gebruiker, de grondstoffen die het gebruikt en gemeenschap waarin het geplaatst wordt. De ontwikkelaar heeft daarmee de taak om kritisch naar zijn productieproces te kijken en waarde toe te voegen met zijn product. [dit is nog vaag, dit gaat eigenlijk misschien niet helemaal hierover?]

\subsection{Het systeem moet om}

te veel focus op de individu
kleine beetjes kleine beetjes
alle problematiek ligt ingebed in een netwerk
neem sojabonen
dus: problemen beschouwen in een netwerk

\subsection{Geld moet een middel zijn om de werkelijke en duurzame economie te dienen}
De afgelopen decennia is geld losgezongen van zijn oorspronkelijke functie: het faciliteren van transacties en de werkelijke economie. Door financialisering – waarbij financiële producten een steeds groter deel van de economische waarde vertegenwoordigen – is de economie losgekomen van de werkelijke waarde en invloed van de onderliggende productieprocessen. Winst en economische groei worden doelen op zichzelf, en geen instrumenten om werkelijke waarde te creëren. De macht van aandeelhouders zorgt voor een korte termijnvisie, waarbij de focus op kwartaalwinsten het bedrijven onmogelijk maakt om op langere termijn duurzame beslissingen te maken. Daarnaast is de geldscheppende functie van Centrale Banken nauwelijks onderdeel van het politieke debat, terwijl de keuzes daarin niet neutraal zijn.

Geld moet dus een middel worden dat de werkelijke economie dient, en moet zo gereguleerd worden dat het duurzame beslissingen aanmoedigt. De functie van geld wordt daarmee weer onderdeel van het politieke debat, in plaats van een neutraal gegeven. Ook is het een correctie op de opgeblazen waarde en beloningen voor sectoren die geen reële waarde toevoegen (denk aan flitshandel) of zelfs waarde vernietigen (fossiele bedrijven).

\subsection{subsidiariteitsbeginsel}
Beslissingen van onderop: subsidiariteitsbeginsel (‘Maak de beslissing zo laag mogelijk, met hen die de gevolgen ervan ervaren, en geef hen ook de macht om die beslissing te nemen’)
Als voorbeelden noemen: Gemeenten, RES (voorzichtig, want veel kritiek door gemeenten op de RES), wooncorporatie, energiecoöperatie

\subsection{Andere maatstaven van succes}
zie recent Helling-artikel Hoe vervangen we het bbp?

Groeiagnostiteit

Consuminderen

Van welvaart naar welzijn

(cultuur en kunst?)

Duurzaamheid, in de breedste zin van sociale, ecologische en economische impact van de organisatie, wordt primair gemeten in termen van financiële winst en verlies en te weinig als maatschappelijk rendement, zoals welzijn voor de mens en gezondheid van onze planeet. Zelfs de maatstaf voor onze welvaart is nog steeds het bruto binnenlands product. Een berekening die geen rekening houdt met negatieve externaliteiten zoals een verhoging van de CO2-uitstoot en/of een hogere inkomensongelijkheid. 

\subsection{Betaal de ware prijs}
De prijs van producten en diensten moet de werkelijke prijs aangeven: niet alleen de kosten die de leverancier heeft gemaakt, maar ook de schade die in het productieproces is toegedaan. [De ware prijs is dus de prijs die je moet betalen voor een product als de sociale en milieukosten bovenop de marktprijs komen.] Hiermee verplaats je de kosten van de schade, van de burger en de lokale overheden, naar de producenten. [rewrite sentence] Zo krijgt de afnemer een eerlijk beeld van de werkelijke belasting op de aarde en worden duurzame opties aantrekkelijker.

Momenteel zijn veel producten nu goedkoper dan verantwoord is, omdat ze de schade bij productie niet meerekenen. [Denk aan een goedkoop vliegticket, vlees waarvoor vee wordt gehouden dat gevoerd wordt met soja waar regenwoud voor gekapt wordt, en een spijkerbroek waarvan de katoenproductie enorm veel water kost en rivieren vergiftigt.] Die schade kan veel vormen aannemen: uitstoot van broeikasgassen, kappen van oerwoud, het bedreigen van de leefomgeving van diersoorten, verlies aan biodiversiteit, verontreiniging van water, lucht en bodem, sociale uitbuiting en het aanwakkeren van conflict in instabiele regio's. Doordat deze schade niet is meegerekend in de prijs die ervoor betaald wordt, is het speelveld ongelijk. Duurzamere en rechtvaardigere opties investeren om deze schade te voorkomen, en zijn daarom duurder en dus minder aantrekkelijk.

Er zijn verschillende instrumenten om het speelveld gelijker te maken ('het beprijzen van negatieve externaliteiten', zoals economen dat noemen). Het belasten van uitstoot met bijvoorbeeld een broeikasgasheffing is daar één van: dan pak je het bij de bron aan. Of een lagere btw-heffing op duurzame producten: dan wordt er bij aankoop vereffent. Belangrijk is dat de methode voor zowel bedrijven als consumenten gelijk werkt. Zo is de energiebelasting nu veel hoger voor consumenten dan voor bedrijven, en worden bedrijven niet gedwongen om hun energieverbruik te verlagen. Een ware prijs geldt voor iedereen.

Natuurlijk is niet alle schade in te prijzen, en wil je vaak voorkomen dat schade überhaupt plaatsvindt. Daarom moet de ware prijs gecombineerd worden met normering: bijvoorbeeld het verplicht stellen van een duurzaamheids- of sociaal keurmerk.

Winst van publieke investering moet terugvloeien naar de samenleving (Rens)
De overheid is de grootste investeerder van Nederland. [TODO: factcheck] Zo investeert ze in infrastructuur, in veiligheid, in gezondheid, in fundamenteel onderzoek, in innovatie, in de rechtsstaat, in sociale zekerheid en in onderwijs. Dat zijn fundamentele drijvers van onze maatschappij. Het creëert een omgeving waarin bedrijvigheid kan floreren: gezonde en goed geschoolde werknemers, juridische zekerheid, nieuwe technologieën en nieuwe markten waar de overheid de wegbereider van is. De private sector profiteert dus volop van de kracht van de overheid.

Tegelijkertijd geeft die sector te weinig terug aan de samenleving die de investeringen heeft opgebracht. Er wordt op creatieve en soms illegale manier zo weinig mogelijk belasting betaald. Bedrijven zitten vaak op of over de grens van toegestane uitstoot- en milieunormen, en proberen dat te verhullen door groene reclamecampagnes. Met patenten houden ze innovatie die in eerste instantie grotendeels door de overheid is gefinancierd achter voor het publiek. In crisistijd weten grote bedrijven en banken precies waar ze hun hand moeten ophouden om gered te worden, om in betere tijden hetzelfde risicovolle gedrag te vertonen waar de crisis mee begon en de bestuurders er rijkelijk voor te belonen.

Kortom: de kosten draagt de samenleving, maar de winst is voor het bedrijfsleven. Als slogan: gesocialiseerde kosten, geprivatiseerde winsten.

We moeten daarom de overheid weer gaan waarderen als investeerder en schepper een goed ondernemersklimaat. Dat betekent dat we erkennen dat de overheid risico mag nemen, en daarin mag falen. En dat de investering van de overheid hoog mogen zijn, omdat de winsten nog hoger zijn en worden verdeeld over de gehele samenleving. Daar staat tegenover dat het bedrijfsleven haar eerlijke deel daarvoor moet betalen met hogere belastingen. En dat ze er niet meer mee weg komt om de regels – over belastingen, concurrentie of milieu – te omzeilen. De 'waakhonden' van de overheid moeten dus ook meer middelen en ruimte krijgen om daarop toe te zien. Dit draagt ook bij aan de maatschappelijke perceptie van de overheid als een ander type ondernemer die risico durft te nemen waar het bedrijfsleven het niet aandurft, en daarvoor uiteindelijk ook beloond wordt.

Een laatste noot: dat bedrijven minder makkelijk gered worden door de overheid mag nooit ten koste gaan van bestaanszekerheid van werknemers en burgers. Dat betekent dat bedrijven nooit 'Too Big Too Fail' mogen zijn, maar ook dat er goede sociale voorzieningen en doorstroommogelijkheden zijn voor mensen die hun baan zijn kwijtgeraakt.

[dit stuk is nogal meerstemmig: gaat het over overheidsinvesteringen a la Mazzucato? Of over het opsplitsen van de grote banken? Of over het vergroten van de waakhond-capaciteit? Het hervormen van het belastingstelsel? Of zijn al die dingen te verenigen onder de noemer 'socialiseren van winsten'?]

\subsection{Overheid is aan zet}
‘De overheid geeft sturing en leiding aan de transitie. Daarmee creëert zij nieuwe mogelijkheden, en bedrijven en organisaties kunnen daarin meekomen’

Zij krijgt ook genoeg controlerende en sturende mogelijkheden (versterken van de waakhonden)

→ zie ook de rol van de overheid in het creëren van nieuwe markten (Mazzucato)

\subsection{Circulariteit}
? misschien 

https://wetenschappelijkbureaugroenlinks.nl/artikel-tijdschrift/tussen-industrie-en-ecosysteem

\subsection{Voorzichtig over nieuwe technologie}

\subsection{Vervuiler betaalt}

\subsection{Een ware transitie vereist een herordening van het eigendoms regime}

Waar ik heen wil is dat naast inkomen en sociale voorzieningen een kapitaal herordening essentieel is willen we werkelijk de sociaal-economische orde veranderen en de transitie draagbaar maken. Ik denk dat de transitie een perfecte manier is om een nieuw eigendoms regime te creëren zonder dat revolutie noodzakelijk is . Het voorbeeld hiervoor is het zweeds sociaal-democratisch model dat tot de jaren 90 erg dichtbij dit ideaal kwam. 

De transitie is zal onvermijdelijk ingrijpend zijn. Er zullen grote offers geleverd moeten brengen, en het is niet meer dan rechtvaardig dat we deze offers zo goed mogelijk proberen te verdelen. Om de transitie dragelijk te maken denken we bijvoorbeeld aan een basisinkomen, gratis onderwijs en zorg, en een gegarandeerde toegang tot noodzakelijke goederen. Hiermee raken we echter niet de kern van het systeem dat ons tot dit punt heeft gebracht. In een samenleving waar slechts een kleine sectie van de bevolking het kapitaal beheert, een samenleving waarin kapitaal zelfs niet meer beheert wordt door een elite, maar een eigen leven leidt in de vorm van instituties die moeten verzamelen, groeien om te overleven is een sterke staat en strenge regels slechts een tijdelijke oplossing voor het klimaat probleem. Zijn betere lonen en gratis diensten slechts een tijdelijk compromis. 
Ware macht ligt in grond, fabrieken, aandelen, mijnen. In het kort, kapitaal. Zonder een grondige herverdeling van wie of wat dit kapitaal bezit blijft de voedingsbodem voor onderdrukking en een nieuwe ecologische crisis.

Voor een hervorming van het eigendoms regime (wie of wat kapitaal bezit, gebaseerd of welke regels) is geen revolutie noodzakelijk. Gezien het tijdsframe van klimaatverandering is een revolutie zelfs niet wenselijk. Wat echter wel kan is dat de transitie zo gestructureerd is dat een de-facto nieuwe eigendoms regime wordt opgezet. In het ideale nieuwe eigendoms regime zijn arbeiders (grotendeels) eigenaar van hun werkplek en zijn de sleutelsectoren (commanding heights of the economy) gezamenlijk democratisch bestuurd eigendom. Een stem over wat er met kapitaal gebeurd kan niet alleen formeel zijn zoals ondernemingsraad. Zonder werkelijk eigendom verwaterd invloed opgelegd door de wet bijna onvermijdelijk. 

Een herziening van het eigendoms regime gebeurt bijvoorbeeld door een verplichte investering in de transitie als onderdeel van het loon voor elke Nederlander. Natuurlijk is een verplichte bijdrage aan een gemeenschappelijk doel qua vorm hetzelfde als een belasting. Het verschil bestaat erin dat in het ene geval bedrijven deels eigendom worden van werknemers en burgers, en in het andere geval de staat het eigendom beheert, en de winst mogelijk uitkeert in sociale diensten. Maar zoals besproken, eigendom in bedrijven gemedieerd door de staat is tijdelijk is tijdelijk zolang de fundamentele logica van ons eigendoms regime blijft staan. 

Een voorbeeld is de Amerikaanse mobilisatie. De Amerikaanse bevolking betaalde bijna de helft van alle kosten van de tweede wereldoorlog middels war bonds. Tijdens de mobilisatie groeide het kapitaal van industrialisten immens. Een eigenaar van een scheepswerf zag zijn kapitaal bijvoorbeeld vertienvoudigen. Voor een periode tijdens en na de oorlog ervaarde amerikanen goede lonen, nieuwe sociale verzieningen. Lonen en voorzieningen die langzaam maar zeker zijn afgebouwd terwijl het kapitaal, de scheepswerf is blijven staan. 

(Achtergrond: inkomens ongelijkheid in nederland is relatief klein. Vermogensongelijkheid is op amerika na het meest extreem in de westerse wereld. Als je daarbij optelt dat bijna al het vermogen van de middenklasse bestaat uit eerste huizenbezit is dit verschil in werkelijkheid nog veel groter. Ten eerste omdat als gevolg van de corona-crisis en de economische crisis die daarop volgt de huizenmarkt zal doen kelderen, ten tweede omdat we in het Deltaplan zo veel mogelijk geld uit de handen van consumenten willen halen, waardoor de markt moeilijk kan herstellen. Ook willen we massaal nieuwe klimaatneutrale huizen bouwen waardoor het tekort, en een van de grootste redenen voor de huizenprijzen oplost. Het weinige kapitaal dat nederlanders hebben wordt uitgevaagd. Het kapitaal dat blijft in huizen is niet liquide, en kan dus niet worden gebruikt voor investeringen in de transitie. Het kapitaal dat bestaat kan dus niet worden gebruikt om hun kapitaal te vergroten. 

\hrule

Opmerkingen
praktische noot: markten blijven bestaan, democratie blijft bestaan, we blijven binnen Europa.

benoemen dat we de staat aanspreken of de overheid?

Nog iets over kernenergie? Over ‘Deep Green’? Ecologisch denken?
