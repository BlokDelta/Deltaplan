\chapter{Andere plannen}

\fullcite{ocasio-cortez_house_2019}
\parencite{ocasio-cortez_house_2019}

% \plancite{hawken_drawdown_2017}
% \begin{itemize}

% 	\item Energie
% 		\begin{itemize}
% 			\item{Wind Turbines (p. 2)}
% 			\item{Solar Farms (p. 8)}
% 			\item{Rooftop Solar (p. 10)}
% 			\item{Micro Grids (p. 5), Grid Flexibility (p. 30), Energy Storage (p.32 \& p.34), Smart Grids (p. 209) }
% 			\item{Biomassa (p. 16), zie ook Perenial Biomass (p. 121)}
% 			\item{Restwarmte (Cogeneration) (p. 22)}
% 			\item{Zonneboilers: Solar water (p. 36)}
% 			\item{Biogas, biovergisters: Methane Digesters (p. 26)}
% 		\end{itemize}

% 	\item Voedsel, biodiversiteit etc.
% 		\begin{itemize}
% 			\item{Plant Rich diet, (p. 38)}
% 			\item{Verminder voedselverspilling (p. 42)}
% 			\item Landbouw en grond (deze secties uitsplitsen of juist niet?)
% 				\begin{itemize}
% 					\item Regenerative Agriculture (p. 54)
% 					\item Silvopasture (p. 50)
% 					\item Intensive Silvopasture(p. 181)
% 					\item Nutrient Management (p. 56)
% 					\item Tree Intercropping (p. 59)
% 					\item Conservation Agriculture (p. 60)
% 					\item Pasture Cropping (p. 175)
% 					\item Permacultuur
% 					\item Marine Permaculture (P. 179) (zie ook Wageningen Kaart)
% 					\item Composting (p. 63)
% 					\item Managed Grazing (p. 72)
% 					\item Ocean Farming (p. 206)
% 					\item minder vee
% 					\item minder mest
% 					\item weg van monoculturen
% 					\item ander akkermanagement (bomenstroken etc.)
% 				\end{itemize}
% 			\item{Green Roofs (p. 90) (ook bij Klimaatadaptatie en Gebouwde Omgeving?)}
% 			\item{Peatlands (p. 122)}
% 		\end{itemize}
% 	\item{\textbf{Internationaal}}
% 	\item{zie stuk Women and Girls (pp. 75 -- 82)}
% 	\item{\textbf{Gebouwde omgeving}}
% 	\item{Net Zero Buildings (p. 84), Smart Thermostats (p. 98)}
% 	\item{Walkable Cities (p. 86), Bike Infrastructure (p. 88)}
% 	\item{Verlichting: Led Lighting (p. 92)}
% 	\item{Warmtepompen: Heat Pumps (p. 94)}
% 	\item{Stadswarmte (p. 99)}
% 	\item{Isolatie: Insulation (p. 101)}
% 	\item{Waterbesparing (p. 170)}
% 	\item{\textbf{Industrie en Materialen}}
% 	\item{Afval: zie Landfill methane (p. 100), Waste to Energy (p. 28)}
% 	\item{Alternatieve materialen: bamboo (p. 117), bioplastic (p. 168), Building With Wood (P. 210)}
% 	\item{Recycling (p. 158) en Industrial Recyling (p. 160) en Recycled Paper (p. 166)}
% 	\item{Alternatief cement (p. 162), zie ook CE Delft/PBL rapporten}
% 	\item{CFK's/koelkasten (p. 164): is dat ook relevant voor Nederland?}
% 	\item{\textbf{Mobiliteit}}
% 	\item{OV: Mass Transit (p. 136)}
% 	\item{Europees treinnetwerk: High Speed Rail (p. 138)}
% 	\item{scheepvaart (p. 140)}
% 	\item{elektrische auto's (p. 142)}
% 	\item{Carpooling (p. 144)}
% 	\item{Elektrische fietsen als vervanger voor auto's: Electric Bikes (p. 146)}
% 	\item{Luchtverkeer (p. 151): opsplitsen in `inperking' en `vergroening'?}
% 	\item{Vrachtwagens: Trucks (p. 152)}

% \end{itemize}


Scrapbook
\begin{itemize}
	\item{Aardwarmte –-- koud-warmteopslag}
	\item{Aardwarmte (diepe, voor kassen etc.)}
	\item{Nul Op De Meter: gecombineerde maatregelen (isolatie, zonnepanelen, warmtepomp, etc.)}

	\item{Herbossing, zie Afforestation (p. 132)}
	\item{`Drawdown': CO2 uit de lucht halen. Zie Direct Air Capture (p. 192), Olivine Stones and way more. See also \cite{climate_cleanup_climate_2020}. Also: vruchtbare aarde als carbon sink.}

\end{itemize}

Soms ook oplossing gecombineerd: steden anders indelen om lopen en fietsen te bevorderen, is zowel sector mobiliteit als sector gebouwde omgeving (of niet?), als sector landgebruik
